\documentclass[svgnames, table, 10pt, aspectratio=169]{beamer} % aspectratio=1610, 149, 54, 43 et 32.]{beamer}
\usepackage{ae,lmodern}
\usepackage[english]{babel}  % package pour langue française
\usepackage[utf8]{inputenc} % compilation des mots accentués
\usepackage[T1]{fontenc}      % césure correcte des mots accentués 
\usepackage{eso-pic,rotating,graphicx} % insertion d'images et manipulation de la position
\usepackage{xcolor}
\usepackage[font=small]{caption}
\usepackage{subcaption} % utilisation d'une légende intermédiaire dans le cas d'une image multiple
\usepackage{url} % utilisation des liens url (utilisation dans le cas de la bibliographie par exemple
\usepackage{tcolorbox,listings} % coloration du texte et possibilité d'inclure blocs de codes
\usepackage{geometry} % customisation des layouts de page (marges, bordures, espacements)
\usepackage{amssymb} % caractères mathématiques
\usepackage{multirow, makecell} % tableau sur plusieurs colonnes fusionnées
\usepackage{bibentry}
\usepackage[backend=biber,style=numeric-comp, sorting=none, natbib=true]{biblatex} % exemples styles : https://fr.overleaf.com/learn/latex/Biblatex_bibliography_styles (mla pas mal)
\usepackage{comment} % utilisation des commentaires de blocs
\setcounter{tocdepth}{2} % profondeur du sommaire (2 = sections, sous sections)
\setcounter{secnumdepth}{1} % profondeur de la numérotation (1 : sections)
\usepackage{ragged2e} % justification du texte dans les itemizes
\usepackage{tikz} % création de schémas directement dans les slides
\usepackage{csquotes} % pour enlever un warning avec biblatex
\usepackage{fontawesome} %emoji
\usepackage{marvosym}
\usepackage[normalem]{ulem}
\usepackage{booktabs}
\usepackage{tabularx}
\usepackage{nth}
\usepackage{siunitx}
\usepackage{pifont}
\newcommand{\cmark}{\ding{51}}
\newcommand{\xmark}{\ding{55}}
% Theme and navigation
\usetikzlibrary{decorations.pathreplacing} % Load the required library

\usetheme[secheader]{Madrid}
\beamertemplatenavigationsymbolsempty
\setbeamertemplate{frametitle continuation}{}
\setbeamertemplate{caption}[numbered]
\setbeamertemplate{headline}


\definecolor{bleuLabSTICC}{RGB}{0,49,99}
\colorlet{beamer@blendedblue}{bleuLabSTICC}

\newcommand{\manualshortsubtitle}{PhD Defense - Lorient}

\makeatletter
\setbeamercolor{footlinecolor}{fg=bleuLabSTICC, bg=white}
\setbeamertemplate{footline}
{
  \leavevmode%
  \hbox{%
  \begin{beamercolorbox}[wd=.55\paperwidth,ht=2.75ex,dp=1ex,left]{footlinecolor}%
    \usebeamerfont{author in head/foot}\insertshortauthor (\insertshortinstitute)
  \end{beamercolorbox}
  \begin{beamercolorbox}[wd=.45\paperwidth,ht=2.75ex,dp=1ex,right]{footlinecolor}%
    \usebeamercolor{date in head/foot}\manualshortsubtitle~--~
    \usebeamerfont{date in head/foot}\insertdate{}\hspace*{2em}
    \insertframenumber{} / \inserttotalframenumber\hspace*{2ex} 
  \end{beamercolorbox}}%
  \vskip0pt%
}
\makeatother

\makeatletter
\setbeamercolor{frametitlecolor}{fg=bleuLabSTICC!75, bg=bleuLabSTICC!10}
\setbeamertemplate{frametitle}{%
  \nointerlineskip
  \begin{beamercolorbox}[sep=.3ex,wd=\paperwidth,leftskip=.5cm,rightskip=0cm]{frametitlecolor}%
    \usebeamerfont{frametitle}\insertframetitle\\[1pt]
  \end{beamercolorbox}
  \ifx\insertframesubtitle\@empty%
  \else
  \nointerlineskip%
  \begin{beamercolorbox}[sep=.3ex,wd=\paperwidth,leftskip=.5cm,rightskip=0cm]{frametitlecolor}%
  \usebeamerfont{framesubtitle}\usebeamercolor[fg]{framesubtitle}\insertframesubtitle%
  \end{beamercolorbox}
  \fi
}
\makeatother

\makeatletter
\setbeamertemplate{title page}{%
    \vfill
    \begingroup
    \centering
    \begin{beamercolorbox}[sep=8pt,center,rounded=true,shadow=false]{title}
        \usebeamerfont{title}\inserttitle\par%
        \ifx\insertsubtitle\@empty%
        \else%
            \vskip0.25em%
            {\usebeamerfont{subtitle}\usebeamercolor[fg]{subtitle}\insertsubtitle\par}%
        \fi%     
    \end{beamercolorbox}%
    \vskip1em\par
    \begin{beamercolorbox}[sep=8pt,center]{author}
        \usebeamerfont{author}\insertauthor
    \end{beamercolorbox}
    \vspace{-.25cm}% NEW
    \begin{beamercolorbox}[sep=8pt,center]{institute}
        \usebeamerfont{institute}\footnotesize\insertinstitute
    \end{beamercolorbox}
    \vspace{-.5cm}% NEW
    \begin{beamercolorbox}[sep=8pt,center]{date}
        \usebeamerfont{date}\footnotesize\insertdate
    \end{beamercolorbox}\vskip0.5em
    \begin{columns}
        \begin{column}{.4\linewidth}
            \centering \scriptsize
            \begin{tabular}{@{}rl@{}}
                % \toprule
                \multicolumn{2}{c}{\textbf{Composition of the Jury}} \\ \midrule
                President of the jury:  & Lionel TORRES          \\
                Reviewers:              & Guillaume BOUFFARD               \\
                                        & Jean-Max DUTERTRE           \\
                Examiners:              & Karine HEYDEMANN        \\
                PhD supervisor:         & Guy GOGNIAT                \\
                PhD co-supervisor:        & Vianney LAP\^OTRE          %\\ \bottomrule
            \end{tabular}
        \end{column}
        \begin{column}{.6\linewidth}
            {\usebeamercolor[fg]{titlegraphic}\inserttitlegraphic\par}
        \end{column}
    \end{columns}
    
    \endgroup
    \vfill
}
\makeatother

\makeatletter
\setbeamertemplate{back page}{%
    \vfill
    \begingroup
    \centering
    \begin{beamercolorbox}[sep=8pt,center,rounded=true,shadow=false]{title}
        \usebeamerfont{title}\inserttitle\par%   
    \end{beamercolorbox}%
    \vskip1em\par
    \begin{beamercolorbox}[sep=8pt,center]{author}
        \usebeamerfont{author}\insertauthor
    \end{beamercolorbox}
    \vspace{-.25cm}% NEW
    \begin{beamercolorbox}[sep=8pt,center]{institute}
        \usebeamerfont{institute}\footnotesize\insertinstitute
        \LARGE \textcolor{bleuLabSTICC!50}{Thank you for your attention.}
    \end{beamercolorbox}
    \vskip0.5em
    \begin{columns}
        \begin{column}{.4\linewidth}
            \centering \scriptsize
            \begin{tabular}{@{}rl@{}}
                % \toprule
                \multicolumn{2}{c}{\textbf{Composition of the Jury}} \\ \midrule
                President of the jury:  & Jean-Max DUTERTRE          \\
                Reviewers:              & Lejla BATINA               \\
                                        & Vincent BEROULLE           \\
                                        & Nele MENTENS               \\
                Examiners:              & Francesco REGAZZONI        \\
                PhD supervisor:         & Guy GOGNIAT                \\
                PhD co-supervisor:        & Vianney LAP\^OTRE          %\\ \bottomrule
            \end{tabular}
        \end{column}
        \begin{column}{.6\linewidth}
            {\usebeamercolor[fg]{titlegraphic}\inserttitlegraphic\par}
        \end{column}
    \end{columns}
    
    \endgroup
    \vfill
}
\makeatother

% Define backpage command
\newcommand{\backpage}{
    \usebeamertemplate{back page}
}

\hypersetup{
  colorlinks=true,
  linkcolor=bleuLabSTICC,
  citecolor=red,
  filecolor=black,
  urlcolor=bleuLabSTICC,
  pdfauthor={Hongwei ZHAO},
  pdftitle={Analysis of vulnerabilities in communication architectures in systems-on-chip with regard to fault attacks and countermeasure propositions for trusted systems},
  pdfsubject={PhD Dissertation Defense},
  pdfkeywords={Embedded systems, Fault injection, Bus architecture, Hardware countermeasures, System-on-Chip security},
  pdfstartview={FitV}
}

\bibliography{bib/bibliographie.bib}
\renewcommand*{\bibfont}{\footnotesize}

\defbeamertemplate{section in toc}{sections numbered roman}{%
  \leavevmode%
  \MakeUppercase{\romannumeral\inserttocsectionnumber}.\ \inserttocsection\par
}
\defbeamertemplate{subsection in toc}{subsections numbered alone}{%
  \leavevmode%
  \hspace{1em}\inserttocsubsectionnumber.\ \inserttocsubsection\par
}

\setbeamertemplate{section in toc}[sections numbered roman]
\setbeamertemplate{subsection in toc}[subsections numbered alone]

\AtBeginSection[]{
  \begin{frame}{}
    \vfill
    \begin{center}
        \Huge \tableofcontents[sectionstyle=show/hide, hideallsubsections]
    \end{center}
    \vfill
  \end{frame}
}

\newcommand{\filigrane}[3]{\begin{tikzpicture}[remember picture,overlay]
	\node [rotate=#2,scale=#3,text opacity=0.75]
	at (current page.center) {\textcolor{red}{\textbf{#1}}};
	\end{tikzpicture}
}

% Define color
\definecolor[named]{LightGray}{RGB}{230,230,230}

\setbeamercovered{transparent}

\newcommand{\tableTwoLines}[2]{\begin{tabular}[c]{@{}c@{}}#1\\ #2\end{tabular}}
\newcommand{\tableCentered}[1]{\begin{tabular}[c]{@{}c@{}}#1\end{tabular}}
\captionsetup{font=footnotesize}
\definecolor{backcolour}{rgb}{0.98,0.98,0.98}
\definecolor[named]{LightGray}{RGB}{230,230,230}

\lstset{
  aboveskip=0.25cm,
  belowskip=0mm,
  basicstyle=\tiny,
  backgroundcolor=\color{backcolour},
  breakatwhitespace=false,
  breaklines=true,
  captionpos=t,
  commentstyle=\color{gray},
  deletekeywords={...},
  escapeinside={\%*}{*)},
  extendedchars=true,
  framexleftmargin=1pt,
  framextopmargin=0pt,
  framexbottommargin=0pt,
  frame=tb,
  keepspaces=true,
  keywordstyle=\color{red},
  language=C,
  literate=
  {²}{{\textsuperscript{2}}}1
  {⁴}{{\textsuperscript{4}}}1
  {⁶}{{\textsuperscript{6}}}1
  {⁸}{{\textsuperscript{8}}}1
  {€}{{\euro{}}}1
  {é}{{\'e}}1
  {è}{{\`{e}}}1
  {ê}{{\^{e}}}1
  {ë}{{\"{e}}}1
  {É}{{\'{E}}}1
  {Ê}{{\^{E}}}1
  {û}{{\^{u}}}1
  {ù}{{\`{u}}}1
  {â}{{\^{a}}}1
  {à}{{\`{a}}}1
  {á}{{\'{a}}}1
  {ã}{{\~{a}}}1
  {Á}{{\'{A}}}1
  {Â}{{\^{A}}}1
  {Ã}{{\~{A}}}1
  {ç}{{\c{c}}}1
  {Ç}{{\c{C}}}1
  {õ}{{\~{o}}}1
  {ó}{{\'{o}}}1
  {ô}{{\^{o}}}1
  {Õ}{{\~{O}}}1
  {Ó}{{\'{O}}}1
  {Ô}{{\^{O}}}1
  {î}{{\^{i}}}1
  {Î}{{\^{I}}}1
  {í}{{\'{i}}}1
  {Í}{{\~{Í}}}1,
  morekeywords={},
  numbers=left,
  numbersep=10pt,
  numberstyle=\tiny\color{black},
  rulecolor=\color{black},
  showspaces=false,
  showstringspaces=false,
  showtabs=false,
  stepnumber=1,
  stringstyle=\color{gray},
  tabsize=4,
  title=\lstname
}

\lstdefinestyle{topPosition}{
  float=ht,
  floatplacement=ht
}

\colorlet{punct}{red!60!black}
% \definecolor{background}{HTML}{EEEEEE}
\definecolor{delim}{RGB}{0,0,255}
\colorlet{numb}{magenta!60!black}
\lstdefinelanguage{json}{
    basicstyle=\tiny,
    numbers=left,
    numberstyle=\tiny,
    stepnumber=1,
    numbersep=8pt,
    showstringspaces=false,
    breaklines=true,
    frame=lines,
    literate=
     *{0}{{{\color{numb}0}}}{1}
      {1}{{{\color{numb}1}}}{1}
      {2}{{{\color{numb}2}}}{1}
      {3}{{{\color{numb}3}}}{1}
      {4}{{{\color{numb}4}}}{1}
      {5}{{{\color{numb}5}}}{1}
      {6}{{{\color{numb}6}}}{1}
      {7}{{{\color{numb}7}}}{1}
      {8}{{{\color{numb}8}}}{1}
      {9}{{{\color{numb}9}}}{1}
      {:}{{{\color{punct}{:}}}}{1}
      {,}{{{\color{punct}{,}}}}{1}
      {\{}{{{\color{delim}{\{}}}}{1}
      {\}}{{{\color{delim}{\}}}}}{1}
      {[}{{{\color{delim}{[}}}}{1}
      {]}{{{\color{delim}{]}}}}{1},
}

\lstdefinelanguage[RISC-V]{Assembler}
{
  alsoletter={.}, % allow dots in keywords
  alsodigit={0x}, % hex numbers are numbers too!
  morekeywords=[1]{ % instructions
    lb, lh, lw, lbu, lhu,
    sb, sh, sw,
    sll, slli, srl, srli, sra, srai,
    add, addi, sub, lui, auipc,
    xor, xori, or, ori, and, andi,
    slt, slti, sltu, sltiu,
    beq, bne, blt, bge, bltu, bgeu,
    j, jr, jal, jalr, ret,
    scall, break, nop, 
    csrr
  },
  morekeywords=[2]{ % sections of our code and other directives
    .align, .ascii, .asciiz, .byte, .data, .double, .extern,
    .float, .globl, .half, .kdata, .ktext, .set, .space, .text, .word, mcycle
  },
  morekeywords=[3]{ % registers
    zero, ra, sp, gp, tp, s0, fp,
    t0, t1, t2, t3, t4, t5, t6,
    s1, s2, s3, s4, s5, s6, s7, s8, s9, s10, s11,
    a0, a1, a2, a3, a4, a5, a6, a7,
    ft0, ft1, ft2, ft3, ft4, ft5, ft6, ft7,
    fs0, fs1, fs2, fs3, fs4, fs5, fs6, fs7, fs8, fs9, fs10, fs11,
    fa0, fa1, fa2, fa3, fa4, fa5, fa6, fa7,
    x0, x1, x2, x3, x4, x5, x6, x7, x8, x9, 
    x10, x11, x12, x13, x14, x15, x16, x17, x18, x19,
    x20, x21, x22, x23, x24, x25, x26, x27, x28, x29,
    x30, x31
  },
  morecomment=[l]{;},   % mark ; as line comment start
  morecomment=[l]{\#},  % as well as # (even though it is unconventional)
  morestring=[b]",      % mark " as string start/end
  morestring=[b]'       % also mark ' as string start/end
}

% usage example:

% define some basic colors
\definecolor{mauve}{rgb}{0.58,0,0.82}

% \lstset{
%   basicstyle=\tiny\ttfamily,                    % very small code
%   breaklines=true,                              % break long lines
%   commentstyle=\itshape\color{green!50!black},  % comments are green
%   keywordstyle=[1]\color{blue!80!black},        % instructions are blue
%   keywordstyle=[2]\color{orange!80!black},      % sections/other directives are orange
%   keywordstyle=[3]\color{red!50!black},         % registers are red
%   stringstyle=\color{mauve},                    % strings are from the telekom
%   identifierstyle=\color{teal},                 % user declared addresses are teal
%   frame=l,                                      % black line on the left side of code
%   language=[RISC-V]Assembler,                   % all code is RISC-V
%   tabsize=4,                                    % indent tabs with 4 spaces
%   showstringspaces=false                        % do not replace spaces with weird underlines
% }
\title{\textsc{Analysis of vulnerabilities in communication architectures in systems-on-chip with regard to fault attacks and countermeasure propositions for trusted systems}}
\subtitle{\textsc{PhD Defense}}
\author[
    Hongwei ZHAO
]{
    \textbf{Hongwei ZHAO}
}
\institute[Université Bretagne Sud, Lab-STICC] % (optional)
{
    {\small Université Bretagne Sud, UMR 6285, Lab-STICC, Lorient, France} \\ \vspace{2mm}
}
\date{December 2, 2025}

% Multiple logos
\titlegraphic{
    \includegraphics[height=1cm]{img/logo/ubs.png}
    \hspace{1cm}
    \includegraphics[height=1cm]{img/logo/labsticc.pdf}
    \hspace{1cm}
    \includegraphics[height=1cm]{img/logo/cnrs.pdf}
}


\begin{document}
% ---------------------------------------------------------------- %
\begin{frame}[plain]
	\titlepage
\end{frame}
% ---------------------------------------------------------------- %
\section*{Introduction}

%%%%%%%%%%%%%%%%%%%%%%%%%%%%%%%%%%%%%%%%%%%%%%%%%%%%%%%%%%%%%%%%%%%%%%%%%%%%
\subsection{Context}
\begin{frame}{Context: Embedded Systems}
    \begin{minipage}[c]{0.45\textwidth}
        \begin{block}{Embedded Systems}
            \begin{itemize}
                \setbeamertemplate{itemize items}[square]
                \justifying
                \item Wide range of applications
                \item Fast growing market
                \item Increasingly vulnerable to multiple threats
            \end{itemize}
        \end{block}
        \begin{figure}
            \centering
            \includegraphics[width=0.7\textwidth]{src/1/img/em.png}
            \caption{Embedded systems (from~\cite{DesignSpark_EmbeddedSystem_2022})}
        \end{figure}
    \end{minipage}\hfill%
    \begin{minipage}[c]{0.55\textwidth}
        \begin{figure}
            \centering
            \includegraphics[width=0.8\textwidth]{src/1/img/market.jpg}
            \caption{Embedded systems market trend (from~\cite{statista_iot})}
            \label{fig:nbr_iot}
        \end{figure}
    \end{minipage}
\end{frame}

%%%%%%%%%%%%%%%%%%%%%%%%%%%%%%%%%%%%%%%%%%%%%%%%%%%%%%%%%%%%%%%%%%%%%%%%%%%%
\begin{frame}{Context: Embedded Systems Under Attack}
    \begin{block}{Threats}
        \begin{itemize}
            \setbeamertemplate{itemize items}[square]
            \justifying
            \item Software attacks: memory overflow~\cite{1467812}, buffer overflow~\cite{821514}, control hijacking, etc.
            \item Hardware attacks: Reverse Engineering~\cite{varady1997reverse}, Side-Channel Attacks~\cite{kelsey1998side}, Fault Injection Attacks~\cite{hsueh1997fault}
        \end{itemize}
    \end{block}

    \begin{center}
        \begin{minipage}[r]{.55\textwidth}
            \begin{figure}
                \centering
                \includegraphics[width=\textwidth]{src/1/img/threats.jpg}
            \end{figure}
        \end{minipage}\hspace{.5cm}%
        \begin{minipage}[c]{0.3\textwidth}
            \captionof{figure}{Possible methods of attacks on embedded systems~\cite{somco-embedded-cybersecurity}}
        \end{minipage}
    \end{center}
\end{frame}

%Comentaire Guy : ajouter un slide pour introduire les attaques en faute et un SoC pour montrer le bus de communication.


%%%%%%%%%%%%%%%%%%%%%%%%%%%%%%%%%%%%%%%%%%%%%%%%%%%%%%%%%%%%%%%%%%%%%%%%%%%%
\subsection{Communication architecture under fault injection attacks}
\begin{frame}{Communication architecture under fault injection attacks}
    \begin{minipage}[c]{0.35\linewidth}
        \begingroup
        \begin{block}{SoC}
            \begin{itemize}
                \item Major realization of embedded systems, also vulnerable to fault attacks \cite{trouchkine2021soc}
                \item Build with different parts, bus as connection and control 
            \end{itemize}
        \end{block}
        \endgroup
    \end{minipage}\hfill%
    \begin{minipage}[c]{0.6\linewidth}
        \begin{figure}
            \centering
            \begin{figure}
                \centering
                \includegraphics[width=0.7\textwidth]{src/1/img/soc.jpg}
            \end{figure}
            \label{fig:schemaDIFT}
            \caption{SoC architecture~\cite{mathworks-soc-architecture}}
        \end{figure}
    \end{minipage}
\end{frame}

\begin{frame}{Communication architecture under fault injection}
    \begin{block}{}
        \begin{itemize}
            \setbeamertemplate{itemize items}[square]
            \justifying
            \item Fault injection attacks: physical modification of data(laser, EM, tension ...) 
            \item Attack on CPU and memory can be solved by integrity mechanisms \cite{burow2017control}
            \item Bus can control the data transfer, vulnerable under fault injection attacks \cite{sun2011design}
        \end{itemize}
    \end{block}
    \begin{figure}
        \centering
        \includegraphics[width=0.4\textwidth]{src/1/img/fia.png}
        \caption{Different injection methods~\cite{fissa}}
    \end{figure}
\end{frame}


%%%%%%%%%%%%%%%%%%%%%%%%%%%%%%%%%%%%%%%%%%%%%%%%%%%%%%%%%%%%%%%%%%%%%%%%%%%%
\subsection{Motivations}
\begin{frame}{Motivations}
    \begin{block}{Vulnerabilities into critical bus}
        \begin{itemize}
            \item Buses are fundamental to SoC functionality
            \item Their standardized and predictable behavior makes them attractive fault-injection targets.
        \end{itemize}
    \end{block}
\end{frame}



%%%%%%%%%%%%%%%%%%%%%%%%%%%%%%%%%%%%%%%%%%%%%%%%%%%%%%%%%%%%%%%%%%%%%%%%%%%%
\subsection{Research challenge}
\begin{frame}{Research challenge}
        \begin{itemize}
            \item Identifying vulnerabilities across different protocols and handshake semantics
            \item Analyzing realistic fault effects at the RTL level within complex interconnects
            \item Designing countermeasures that remain effective without incurring prohibitive hardware overhead
        \end{itemize}
\end{frame}
%%%%%%%%%%%%%%%%%%%%%%%%%%%%%%%%%%%%%%%%%%%%%%%%%%%%%%%%%%%%%%%%%%%%%%%%%%%%
\subsection{Objectives}
\begin{frame}{Objectives of this PhD Thesis}
    \begin{block}{}
        \begin{itemize}
            \setbeamertemplate{itemize items}[triangle]
            \justifying
            \item Analyze the behavior of Wishbone, AXI-Lite, and AXI under fault injection attacks
            \item Evaluate existing hardware and software countermeasures to determine their practical limitations
            \item Propose and validate a hardware-integrated countermeasure that can detect bus-level faults
        \end{itemize}
    \end{block}
\end{frame}
%%%%%%%%%%%%%%%%%%%%%%%%%%%%%%%%%%%%%%%%%%%%%%%%%%%%%%%%%%%%%%%%%%%%%%%%%%%%
%
% ---------------------------------------------------------------- %
\begin{frame}{Outline}
	\begin{columns}
		\begin{column}{.15\linewidth}
			\hfill
		\end{column}
		\begin{column}{.8\linewidth}
			\Large\tableofcontents[hideallsubsections]
		\end{column}
		\begin{column}{.1\linewidth}
			\hfill
		\end{column}
	\end{columns}
\end{frame}
%
% ---------------------------------------------------------------- %
\section{Experimental Setup}

%%%%%%%%%%%%%%%%%%%%%%%%%%%%%%%%%%%%%%%%%%%%%%%%%%%%%%%%%%%%%%%%%%%%%%%%%%%%
\subsection{Construction of the SoC}

\begin{frame}{LiteX Framework}
    \begin{columns}
        \begin{column}{0.45\textwidth}
        \begin{itemize}
            \item Open-source SoC builder for FPGA-based systems \cite{litex}.
            \item Modular design with Python-based HDL (Migen).
            \item Supports several processor architectures (e.g. RISC-V).
            \item No built-in security → interesting to explore vulnerability analysis in this functional-oriented architecture.
        \end{itemize}
    \end{column}
    \begin{column}{0.55\textwidth}
    \begin{figure}
        \centering
        \includegraphics[width=0.7\textwidth]{src/2/img/litex.png}
        \caption{LiteX framework components}
    \end{figure}    
    \end{column}
    \end{columns}
\end{frame}

%%%%%%%%%%%%%%%%%%%%%%%%%%%%%%%%%%%%%%%%%%%%%%%%%%%%%%%%%%%%%%%%%%%%%%%%%%%%
\subsection{SoC Configuration}
\begin{frame}{SoC Architecture Overview}
    \begin{itemize}
        \item CPU: VexRiscv (RISC-V ISA).
        \item Interconnect protocols: Wishbone, AXI-Lite, AXI.
        \item Memory regions: ROM, SRAM, CSR, MAIN\_RAM.
        \item Target FPGA: Digilent Basys3 (Artix-7).
    \end{itemize}
\begin{figure}
    \centering
    \includegraphics[width=0.4\textwidth]{src/2/img/socgeneral.png}
    \caption{Our SoC architecture}
\end{figure}
\end{frame}

%%%%%%%%%%%%%%%%%%%%%%%%%%%%%%%%%%%%%%%%%%%%%%%%%%%%%%%%%%%%%%%%%%%%%%%%%%%%
\subsection{Benchmark}
\begin{frame}{VerifyPin Benchmark}
    \begin{itemize}
        \item Written in C \cite{DureuilPPLCC16}.
        \item Simulates a PIN verification process.
        \item Suite of 8 implementations: V0 (unprotected) + V1–V7 (protected).
    \end{itemize}
\end{frame}

\begin{frame}{VerifyPin V0 Example}
    \begin{columns}
        \begin{column}{0.5\textwidth}
        \begin{itemize}
            \item Compares user PIN (\texttt{"0000"}) with card PIN (\texttt{"4321"}).
            \item Fault success: \texttt{g\_authenticated} set to 1 despite mismatch.
        \end{itemize}
        \end{column}
    \begin{column}{0.5\textwidth}
        \begin{figure}
        \centering
        \includegraphics[width=0.9\textwidth]{src/2/img/v0.png}
        \caption{C code of VerifyPin function in benchmark V0}
        \end{figure}
    \end{column}
    \end{columns}
\end{frame}

\begin{frame}{Countermeasures implemented in V1 to V7}
\begin{itemize}
  \item HB: Hardened Boolean
  \item FTL: Fixed-Time Loop
  \item INL: Inlined Function
  \item DPTC/PTCBK: Token Counter decremented first/Back up
  \item LC: Loop Counter
  \item DC/DT: Double Call/Test
  \item SC: Step Counter
\end{itemize}
\end{frame}

\begin{frame}{VerifyPin V1 Example}
    \begin{columns}
        \begin{column}{0.5\textwidth}
        \begin{itemize}
            \item Implement with Hardened Boolean.
            \item Replace 1 and 0 with BOOL\_True(0xAA) and BOOL\_False(0x55).
            \item Detect fault => execute countermeasure()
        \end{itemize}
        \end{column}
    \begin{column}{0.5\textwidth}
        \begin{figure}
        \centering
        \includegraphics[width=0.9\textwidth]{src/2/img/v1.png}
        \caption{C code of VerifyPin function in benchmark V1}
        \end{figure}
    \end{column}
    \end{columns}
\end{frame}
%%%%%%%%%%%%%%%%%%%%%%%%%%%%%%%%%%%%%%%%%%%%%%%%%%%%%%%%%%%%%%%%%%%%%%%%%%%%
\subsection{FISSA Tool}
\begin{frame}{FISSA Overview}
    \begin{itemize}
        \item Python-based tool for fault injection campaigns \cite{fissa}.
        \item Works with HDL simulators (Questasim, Vivado, Verilator).
        \item Automates TCL script generation and simulation logging.
    \end{itemize}
\begin{figure}
    \centering
    \includegraphics[width=0.8\textwidth]{src/2/img/fissa.png}
    \caption{FISSA components}
\end{figure}
\end{frame}

\begin{frame}{Register lists}
    \begin{itemize}
        \item Non-control registers (I/O, LEDs, clock/reset, timers, program data) were excluded due to low injection effectiveness.
        \item Fault injection was applied to Wishbone, AXI-Lite, and AXI under multiple fault models.
    \end{itemize}
        
    \begin{table}
    \centering
    \caption{All the registers targeted by fault injection in the 3 buses}
    \label{inject reg}
    \resizebox{\textwidth}{!}{
    \begin{tabular}{|l|l|}
    \hline
    Bus & Registers \\
    \hline
    Wishbone & ACK, SEL, done, grant \\
    AXI-Lite & state, selection driver, last\_was\_read, rr\_read\_grant, completion flags \\
    AXI & same with AXI-Lite, ax\_beat\_first, ax\_beat\_last, last\_ar\_aw\_n, pipe\_valid\_source \\
    \hline
    \end{tabular}}
    \end{table}
\end{frame}

\begin{frame}{Considered Fault Models}
    \begin{enumerate}
        \item Bit-Flip: single bit in register.
        \item Manipulate Register: arbitrary bit-flips in one register.
        \item 2 Bit-Flips: exactly two bits flipped.
        \item Manipulate Two Registers: arbitrary bit-flips in two registers.
    \end{enumerate}
\begin{figure}
    \centering
    \includegraphics[width=0.8\textwidth]{src/2/img/fault model.png}
    \caption{Illustration of our fault models}
\end{figure}
\end{frame}

\begin{frame}{Outcome Categories}
    \begin{itemize}
        \item Crash: simulation terminated.
        \item Software detection: software countermeasure to detect the fault.
        \item Hardware detection: hardware countermeasure to detect the fault.
        \item Hardware correction: hardware countermeasure to correct the fault.
        \item Success: unauthorized authentication achieved.
        \item Change: memory state altered.
        \item Silence: no visible impact.
    \end{itemize}
\end{frame}
%%%%%%%%%%%%%%%%%%%%%%%%%%%%%%%%%%%%%%%%%%%%%%%%%%%%%%%%%%%%%%%%%%%%%%%%%%%%
\subsection{Fault Models}
\begin{frame}{Fault Injection Campaign}
    \begin{itemize}
        \item Attacker has no PIN knowledge.
        \item Attacker has knowledge of the bus architecture.
        \item Faults injected at the RTL level via simulation.
        \item Target: control registers connected to the system bus.
    \end{itemize}
\end{frame}
%
% ---------------------------------------------------------------- %
\section{Vulnerabilities Exploitation on Bus}

%%%%%%%%%%%%%%%%%%%%%%%%%%%%%%%%%%%%%%%%%%%%%%%%%%%%%%%%%%%%%%%%%%%%%%%%%%%%
\subsection{Wishbone Protocol}
\begin{frame}{Handshake in Wishbone}
    \begin{columns}
        \begin{column}{0.4\textwidth}
            \begin{itemize}
                \item Acknowledge generation: ack register controls acknowledge signal in CPU
                \item Selection mechanism: selection registers selects memory 
            \end{itemize}
        \end{column}
    \begin{column}{0.6\textwidth}
        \begin{figure}
            \centering
            \includegraphics[width=0.7\textwidth]{src/3/img/wishbone bus.png}
            \caption{ACK and SEL register connections on Wishbone bus}
        \end{figure}
    \end{column}
    \end{columns}    
\end{frame}

%%%%%%%%%%%%%%%%%%%%%%%%%%%%%%%%%%%%%%%%%%%%%%%%%%%%%%%%%%%%%%%%%%%%%%%%%%%%
\subsection{AXI-Lite and AXI}
\begin{frame}{Handshake in AXI-Lite and AXI}
    \begin{columns}
        \begin{column}{0.4\textwidth}
            \begin{itemize}
                \item Acknowledge generation: state-machine controls acknowledge signal in CPU
                \item Selection mechanism: state-machine and selection register assign in selection signal to select memory 
            \end{itemize}
        \end{column}
    \begin{column}{0.6\textwidth}
        \begin{figure}
            \centering
            \includegraphics[width=0.7\textwidth]{src/3/img/axi.png}
            \caption{Connection of handshake and selection signals in AXI and AXI-Lite bus protocols}
        \end{figure}
    \end{column}
    \end{columns} 
\end{frame}

%%%%%%%%%%%%%%%%%%%%%%%%%%%%%%%%%%%%%%%%%%%%%%%%%%%%%%%%%%%%%%%%%%%%%%%%%%%%
\subsection{Fault Injection Result}
\begin{frame}{Fault Injection Result}
\begin{table}
\centering
\caption{Fault injection results for each bus and fault model in V0}
\label{Vunerabilities on 3 buses}
\begin{tabular}{llrrrr}
\toprule
bus & fault model & crash & success & change & silence \\
\midrule
& Bit-Flip & 47 & 37 & 124 & 2366 \\
& Manipulate Register   & 320 & 43 & 161 & 6496 \\
& 2 Bit-Flips & 547 & 272 & 914 & 11137 \\
\multirow{-4}{*}{Wishbone}  & Manipulate Two Registers & 5178 & 778 & 2891 & 63225 \\
\midrule
& Bit-Flip & 282 & 4 & 1913 & 11577 \\
& Manipulate Register & 391 & 6 & 2589 & 29942 \\
& 2 Bit-Flips & 10725 & 153 & 69811 & 194831 \\
\multirow{-4}{*}{AXI-Lite}  &  Manipulate Two Registers & 36215 & 549 & 224347 & 1236105 \\
\midrule
& Bit-Flip & 158 & 4 & 4833 & 34269 \\
& Manipulate Register   & 435 & 6 & 6723 & 90996 \\
& 2 Bit-Flips & 14760 & 333 & 428382 & 1421565 \\
\multirow{-4}{*}{AXI} &  Manipulate Two Registers & 104655 & 1328 & 1514455 & 9762850 \\         
\bottomrule  
\end{tabular}
\end{table}
\end{frame}

%%%%%%%%%%%%%%%%%%%%%%%%%%%%%%%%%%%%%%%%%%%%%%%%%%%%%%%%%%%%%%%%%%%%%%%%%%%%
\subsection{Result Analysis}

\begin{frame}{Analysis on Wishbone}
    \begin{columns}
        \begin{column}{0.4\textwidth}
            \begin{itemize}
                \item ACK register is the most frequently targeted in successful attacks across all fault models.
            \end{itemize}
        \end{column}
    \begin{column}{0.6\textwidth}
        \begin{table}
        \centering
        \caption{Distribution of successful register combination attacks under different fault models on the Wishbone Bus}
        \label{Percentage on Wishbone}
          \resizebox{\columnwidth}{!}{
        \begin{tabular}{lcccc}
        \hline
        Fault model & \texttt{ack} & \texttt{sel} & \texttt{ack} \& \texttt{grant} & \texttt{ack} \& \texttt{sel} \\ 
        \hline
        Bit-flip & 94.59\% & 5.41\% & - & - \\
        Manipulate Register & 81.40\% & 18.60\% & - & - \\
        2 Bit-Flips & 94.12\% & 3.68\% & 1.10\% & 1.10\% \\
        Manipulate Two Registers & 85.73\% & 12.34\% & 0.38\% & 1.55\% \\
        \hline
        \end{tabular}
        }
        \end{table}
    \end{column}
    \end{columns} 
\end{frame}

\begin{frame}{Analysis on AXI-Lite}
    \begin{columns}
        \begin{column}{0.35\textwidth}
            \begin{itemize}
                \item State register is the primary target for successful attacks across all fault models.
            \end{itemize}
        \end{column}
    \begin{column}{0.65\textwidth}
        \begin{table}
        \centering
        \caption{Distribution of successful register combination attacks under different fault models on the AXI-Lite Bus}
        \label{Percentage on Lite}        
        \begin{tabularx}{\textwidth}{|l|X|X|X|}
        \hline
        Fault model & state & selection driver \& state & completion flag \& state \\
        \hline
        Bit-flip & 100.00\% & - & - \\
        Manipulate Register & 100.00\% & - & - \\
        2 Bit-Flips & 98.02\% & 1.32\% & 0.66\% \\
        Manipulate Two Registers & 98.36\% & 1.46\% & 0.18\% \\
        \hline
        \end{tabularx}        
        \end{table}
    \end{column}
    \end{columns}
\end{frame}

%%%%%%%%%%%%%%%%%%%%%%%%%%%%%%%%%%%%%%%%%%%%%%%%%%%%%%%%%%%%%%%%%%%%%%%%%%%%
\begin{frame}{Fault effect}
\begin{table}
\centering
\label{compare bus}

\begin{tabularx}{\textwidth}{|l|X|X|X|}
\hline
                                          & Wishbone         & AXI-Lite         & AXI          \\
\hline                                          
\multirow{3}{*}{Bit-flip}                 & instruction skip & data reset       & data reset   \\
                                          & data reset       &                  & data misread \\
                                          & data multiread   &                  &              \\
\hline                                          
\multirow{4}{*}{Manipulate Register}      & instruction skip & data reset       & data reset   \\
                                          & data reset       & data misread     & data misread \\
                                          & data misread     &                  &              \\
                                          & data multiread   &                  &              \\
\hline                                          
\multirow{4}{*}{2 Bit-Flips}              & instruction skip & instruction skip & data reset   \\
                                          & data reset       & data reset       & data misread \\
                                          & data misread     & data misread     &              \\
                                          & data multiread   & data multiread   &              \\
\hline                                          
\multirow{4}{*}{Manipulate Two Registers} & instruction skip & data reset       & data reset   \\
                                          & data reset       & instruction skip & data misread \\
                                          & data misread     & data misread     &              \\
                                          & data multiread   & data multiread   &             \\
\hline                                          
\end{tabularx}

\end{table}
\end{frame}

\begin{frame}{Wishbone: Success Distribution}
    \begin{columns}
        \begin{column}{0.45\textwidth}
            \begin{itemize}
                \item Instruction- vs. data-related fault ratios under “Manipulate Two Registers”.
                \item AXI-Lite/AXI: Frequent data resets (to 0) can unintentionally match the user PIN, making data corruption the main attack pathway.
            \end{itemize}
        \end{column}
    \begin{column}{0.55\textwidth}
        \begin{table}
        \centering
        \caption{Percentage of data or instruction vulnerabilities by manipulate 2 registers attacks across the three bus architectures}
        \label{Percentage data}
        \begin{tabularx}{\textwidth}{|l|X|X|X|}
        \hline
        & Wishbone & AXI-Lite & AXI \\
        \hline
        Data related & 17.22\% & 97.27\% & 100\% \\
        Instruction related & 82.78\% & 2.73\% & 0\% \\
        \hline
        \end{tabularx}
        \end{table}
    \end{column}
    \end{columns}
\end{frame}
%
% ---------------------------------------------------------------- %
\section{Protection Implementation on Wishbone Bus}

\subsection{Wishbone Vulnerabilities}
\begin{frame}{Critical Signals Under Attack}
    \begin{itemize}
        \item \texttt{ack}: 4-bit register, controlled by sel register, cyc/stb signal, assign to ack\_d and ack\_i
        \item \texttt{sel}: 4-bit register, assigned by adr register
        \item \texttt{grant}: arbitration register for ack\_d/ack\_i, etc
        
    \end{itemize}

    \begin{columns}
        \begin{column}{0.55\textwidth}
        \begin{figure}
            \centering
            \includegraphics[width=0.7\textwidth]{src/3/img/wishbone bus.png}
        \end{figure}
        \end{column}
    \begin{column}{0.45\textwidth}
        \begin{table}
        \centering
        \caption{Distribution of successful register combination attacks under different fault models on the Wishbone Bus}
        \label{Percentage on Wishbone}
          \resizebox{\columnwidth}{!}{
        \begin{tabular}{lcccc}
        \hline
        Fault model & \texttt{ack} & \texttt{sel} & \texttt{ack} \& \texttt{grant} & \texttt{ack} \& \texttt{sel} \\ 
        \hline
        Bit-flip & 94.59\% & 5.41\% & - & - \\
        Manipulate Register & 81.40\% & 18.60\% & - & - \\
        2 Bit-Flips & 94.12\% & 3.68\% & 1.10\% & 1.10\% \\
        Manipulate Two Registers & 85.73\% & 12.34\% & 0.38\% & 1.55\% \\
        \hline
        \end{tabular}
        }
        \end{table}
    \end{column}
    \end{columns} 
    
\end{frame}

\begin{frame}{ACK Under Attack}
    \begin{columns}
        \begin{column}{0.4\textwidth}
          \textbf{Normal behavior:}
            \begin{itemize}
              \item \texttt{ack1} toggles each cycle while other \texttt{ack} signals remain low.
              \item \texttt{ack\_d} updates both address and data every two cycles.
              \item CPU sequentially reads values and commits them to cache.
            \end{itemize}
        \end{column}
    \begin{column}{0.6\textwidth}
        \begin{figure}
            \centering
            \includegraphics[width=0.9\textwidth]{src/4/img/nofault1.png}
            \caption{Impact of acknowledge signal on address, data signals and CPU cache without an attack}
        \end{figure}
    \end{column}
    \end{columns} 
\end{frame}

\begin{frame}{ACK Under Attack}
    \begin{columns}
        \begin{column}{0.4\textwidth}
          \textbf{Faulted behavior:}
            \begin{itemize}
              \item A bit-flip on \texttt{ack0} disrupts the expected read timing.
              \item Two-cycle transfers collapse into a single cycle, causing premature cache allocation.
              \item Cache lines are filled with corrupted values (e.g., PIN replaced by zeros).
              \item The program compares identical PINs and incorrectly grants access.
              \item It corresponds to an Instruction skip
            \end{itemize}
        \end{column}
        \begin{column}{0.6\textwidth}
            \begin{figure}
                \centering
                \includegraphics[width=0.9\textwidth]{src/4/img/fault1.png}
                \caption{Impact of acknowledge signal on address, data signals and CPU cache with an attack}
            \end{figure}
        \end{column}
    \end{columns} 
\end{frame}


%%%%%%%%%%%%%%%%%%%%%%%%%%%%%%%%%%%%%%%%%%%%%%%%%%%%%%%%%%%%%%%%%%%%%%%%%%%%
\subsection{Software Countermeasures}

\begin{frame}{Countermeasure Deployment}
\begin{table}
\centering
  \caption{Software countermeasures deployed in each benchmark version}
  \label{tab:benchmark cm}
  \begin{tabular}{cccccccccc}
    \hline
    & HB & FTL & INL & DPTC & PTCBK & LC & DC & DT & SC\\
    \hline
    \texttt{V0} & & & & & & & & &  \\
    \texttt{V1} & \checkmark & & & & & & & &  \\
    \texttt{V2} & \checkmark & \checkmark & & & & & & & \\
    \texttt{V3} & \checkmark & \checkmark & \checkmark & & & & & & \\
    \texttt{V4} & \checkmark & \checkmark & \checkmark & \checkmark & \checkmark & \checkmark & & &\\
    \texttt{V5} & \checkmark & \checkmark & & \checkmark & & & \checkmark & &  \\
    \texttt{V6} & \checkmark & \checkmark & \checkmark & \checkmark & & & & \checkmark &  \\
    \texttt{V7} & \checkmark & \checkmark & \checkmark & \checkmark & & & & \checkmark & \checkmark\\
    \hline
  \end{tabular}
\end{table}
\end{frame}

%%%%%%%%%%%%%%%%%%%%%%%%%%%%%%%%%%%%%%%%%%%%%%%%%%%%%%%%%%%%%%%%%%%%%%%%%%%%
\subsection{Software Countermeasure Results}
\begin{frame}{Fault Injection Results et Resource Analysis Lizard (Software)}
\begin{itemize}
  \item 8 versions of the benchmark with/without software countermeasure
  \item Fault model: Bitflip
  \item Lizard: analysis NLOC, CCN, token
\end{itemize}
\begin{table}
\captionsetup{skip=10pt}
  \centering
  \label{tab:benchmark result}
  \resizebox{\textwidth}{!}{%
\begin{tabular}{cccccccccccc}
\hline
& crash & detect & success & change & silence & success rate & detect rate & sum & NLOC & CCN & token \\
\hline
V0 & 47 & 0 & 37 & 124 & 2366 & 1.44\% & 0 & 2574 & 32 & 6 & 107 \\
V1 & 61 & 21 & 36 & 136 & 3266 & 1.02\% & 0.60\% & 3520 & 36 & 7 & 127 \\
V2 & 98 & 20 & 30 & 164 & 5078 & 0.56\% & 0.37\% & 5390 & 44 & 8 & 149 \\
V3 & 58 & 19 & 31 & 96 & 3789 & 0.78\% & 0.48\% & 3993 & 32 & 7 & 129 \\
V4 & 92 & 101 & 33 & 104 & 5478 & 0.57\% & 1.74\% & 5808 & 47 & 11 & 191 \\
V5 & 105 & 28 & 9 & 153 & 5238 & 0.16\% & 0.51\% & 5533 & 42 & 9 & 163 \\
V6 & 61 & 49 & 14 & 108 & 4377 & 0.30\% & 1.06\% & 4609 & 38 & 9 & 153 \\
V7 & 105 & 186 & 9 & 252 & 6708 & 0.12\% & 2.56\% & 7260 & 77 & 18 & 312\\       
\hline                          
\end{tabular}
}
\end{table}
\end{frame}

\begin{frame}{Software Countermeasures Overview}
  Instruction protection:
\begin{itemize}
  \item HB: Prevents register values defaulting to 0/1, mitigating some branch instruction.
  \item FTL/LC: Fixes/Records  loop iteration count, blocking loop manipulation.
  \item INL: Merges functions can't reduce fault.
  \item DPTC/PTCBK: State counters not targeted, no effective defense observed.
  \item DC/DT: Redundant execution neutralizes function-call attacks.
  \item SC: Detects skipped instructions, effective against instruction-skipping.
\end{itemize}
  Data protection: Only INL (fewer reads) and DC (double read) provide defense.
\begin{table}
  \caption{Evaluation of software countermeasures against fault attacks on instructions and data}
  \label{tab:cm evaluate}
  \resizebox{\columnwidth}{!}{
\begin{tabular}{lccccccccc}
\hline
& HB & FTL & INL & DPTC & PTCBK & LC & DC & DT & SC \\
\hline
Instruction fault & \cmark & \cmark & \xmark & \xmark & \xmark & \cmark & \cmark & \cmark & \cmark \\
Data fault        & \xmark & \xmark & \cmark & \xmark & \xmark & \xmark & \cmark & \xmark & \xmark\\
\hline
\end{tabular}
}
\end{table}
\end{frame}

%%%%%%%%%%%%%%%%%%%%%%%%%%%%%%%%%%%%%%%%%%%%%%%%%%%%%%%%%%%%%%%%%%%%%%%%%%%%
\subsection{Hardware Countermeasures}
\begin{frame}{Hardware Countermeasures Overview}
\begin{columns}
  \column{0.5\textwidth}
  \begin{itemize}
    \item Software-only defenses are limited, Hardware countermeasures are needed.
    \item Architectural change: combi=> mux, reducing multiple-read attacks.
    \item Hardware protections on \texttt{ack} and \texttt{sel} registers.
  \end{itemize}

  \column{0.5\textwidth}
  \begin{figure}
    \centering
    \includegraphics[width=\textwidth]{src/4/img/change arch.png}
    \caption{Optimized connection between the selection register and the memory using multiplexers}
  \end{figure}
\end{columns}
\end{frame}

\begin{frame}{Hardware Countermeasures Overview}
\begin{itemize}
\item Simple Parity: Detects faults using a 1-bit parity code.
\item Duplication: Creates a duplicate of the registers and compares it with the unprotected version. 
\item Complementary Duplication: Duplicates the inverse of the registers and compares it with the unprotected version.
\item Hamming Code: Corrects the output signal of a register using a 3-bit checksum.
\item SECDED Code: Corrects or detects errors using a 4-bit checksum.
\item Triplication: Duplicates a register twice, correcting the signal if two registers have matching outputs and detecting errors if all three differ.
\end{itemize}
\end{frame}

%%%%%%%%%%%%%%%%%%%%%%%%%%%%%%%%%%%%%%%%%%%%%%%%%%%%%%%%%%%%%%%%%%%%%%%%%%%%
\begin{frame}{Results of Hardware Countermeasure with VerifyPin V0}
\begin{table}
\captionsetup{skip=10pt}
\centering
  \label{tab:v0 under attack result}
\scalebox{0.6}{
\begin{tabular}{llccc}
\hline
Countermeasure & Fault model & Success rate & Detection rate & Correction rate \\
\hline
 & Bit-Flip & 1.44\% & - & - \\
 & Manipulate Register & 0.61\% & - & - \\
 & 2 Bit-Flips & 2.11\% & - & - \\
\multirow{-4}{*}{Unprotected} & Manipulate Two Registers  & 1.08\%  & - & - \\
\hline
 & Bit-Flip & 0\% & 69.54\% & - \\
 & Manipulate Register & 0.04\% & 34.77\% & - \\
 & 2 Bit-Flips & 0.89\% & 64.38\% & - \\
\multirow{-4}{*}{Simple parity} & Manipulate Two Registers & 0.27\% & 50.34\% & - \\
\hline
 & Bit-Flip & 0\% & 77.66\% & - \\
 & Manipulate Register & 0\% & 44.94\% & - \\
 & 2 Bit-Flips & 0.15\% & 86.43\% & - \\
\multirow{-4}{*}{Duplication} & Manipulate Two Registers & 0.04\% & 66.46\% & - \\
\hline
 & Bit-Flip & 0\% & 77.66\% & - \\
Complementary & Manipulate Register & 0\% & 44.94\% & - \\
Duplication & 2 Bit-Flips & 0.15\% & 86.43\% & - \\
 & Manipulate Two Registers & 0.04\% & 66.46\% & - \\
\hline
 & Bit-Flip & 0\% & - & 80\% \\
 & Manipulate Register & 0.49\% & - & 58.16\% \\
 & 2 Bit-Flips & 0.97\% & - & 91.30\% \\
\multirow{-4}{*}{Hamming code} & Manipulate Two Registers & 1.01\% & - & 75.95\% \\
\hline
 & Bit-Flip & 0\% & 0\% & 85.71\% \\
 & Manipulate Register & 0\% & 0\% & 81.82\% \\
 & 2 Bit-Flips & 0.23\% & 20\% & 76.94\% \\
\multirow{-4}{*}{Triplication} & Manipulate Two Registers & 0.14\% & 36.74\% & 58.97\% \\
\hline
 & Bit-Flip & 0\% & 11.77\% & 70.59\% \\
 & Manipulate Register & 0.32\% & 34\% & 37.92\% \\
 & 2 Bit-Flips & 0\% & 45.56\% & 52.06\% \\
\multirow{-4}{*}{Secded code} & Manipulate Two Registers & 0.45\% & 51.94\% & 36.19\% \\
\hline
\end{tabular}
}
\end{table}
\end{frame}

% \begin{frame}{Hardware Countermeasures Overview}
% \begin{itemize}
%   \item Duplication / complementary duplication: lower success rates, higher detection probability.
%   \item SECDED (Hamming code): strong correction, effective against dual-bit faults across registers.
%   \item Triplication: best for single-bit errors and localized multi-bit corruption.
%   \item All methods defend against single bit-flip.
%   \item Duplication, complementary duplication, triplication: effective for multi-bit faults in one register.
%   \item SECDED: defends against two bit-flips.
%   \item No method fully prevents multi-register multi-bit faults; triplication performs better.
%   \item Vulnerabilities:
%     \begin{itemize}
%       \item Simple parity: weak against two bit-flips (parity + signal line).
%       \item SECDED: vulnerable to multi-register manipulations, possible false corrections.
%       \item Duplication / complementary duplication / triplication: susceptible to two bit-flips mirroring errors.
%     \end{itemize}
% \end{itemize}
% \end{frame}

%%%%%%%%%%%%%%%%%%%%%%%%%%%%%%%%%%%%%%%%%%%%%%%%%%%%%%%%%%%%%%%%%%%%%%%%%%%%
\subsection{Hardware Resource Overhead}
\begin{frame}{Resource Overhead Analysis}
\begin{itemize}
  \item Countermeasures increase LUT usage by max. 0.7\%.
  \item Frequency reduced by up to 0.97\%.
  \item Differences largely due to synthesizer auto-optimization.
  \item Overall: negligible additional hardware resource loss.
\end{itemize}

\begin{table}
    \centering
  \caption{Hardware resource overhead of each hardware countermeasure (LUT, Flip-Flop, frequency)}
  \label{tab:countermeasures synthesis}
\begin{tabular}{cccc}
\hline
Countermeasure & LUT & Flip-Flops & frequency (MHz) \\
\hline
Unprotected & 2198 & 1793 & 70.13 \\
Simple parity & 2214 & 1791 & 70.27 \\
Duplication & 2201 & 1791 & 70.18 \\
Complimentary & 2199 & 1791 & 70.37 \\
Hamming code & 2199 & 1794 & 70.32 \\
Triplication & 2199 & 1791 & 70.27 \\
Secded code & 2193 & 1789 & 69.44 \\
\hline
\end{tabular}
\end{table}
\end{frame}

%%%%%%%%%%%%%%%%%%%%%%%%%%%%%%%%%%%%%%%%%%%%%%%%%%%%%%%%%%%%%%%%%%%%%%%%%%%%
\subsection{Compare Countermeasures}
\begin{frame}{Comparison of the Protection Effectiveness}
\begin{itemize}
  \item Hardware-only protections achieve consistently lower attack success rates.
  \item In some fault models, hardware countermeasures reduce success rate to zero.
\end{itemize}

    \begin{columns}
        \begin{column}{0.5\textwidth}
        \begin{figure}
          \centering
          \includegraphics[width=0.7\linewidth]{src/4/img/hw rate.png}
          \caption{Success rates under the four fault models for the benchmark V0 with different hardware countermeasures}
          \label{hw cmp}
        \end{figure}
        \end{column}
    \begin{column}{0.5\textwidth}
        \begin{figure}[H]
          \centering
          \includegraphics[width=0.7\linewidth]{src/4/img/sw rate.png}
          \caption{Success rates under the four fault models for the seven benchmark versions with software countermeasures}
          \label{sw cmp}
        \end{figure}
    \end{column}
    \end{columns} 
\end{frame}

\begin{frame}{Comparison of the Protection Effectiveness}
    \begin{columns}
        \begin{column}{0.5\textwidth}
        \begin{itemize}
          \item Figure~\ref{hwsw combine}: clear improvement with duplication + software vs. duplication alone.
          \item No hardware/software combination fully neutralizes the Manipulate Two Registers fault model.
          \item Persistent vulnerability indicates need for more advanced or hybridized protection beyond duplication/redundancy.
        \end{itemize}
        \end{column}
    \begin{column}{0.5\textwidth}
        \begin{figure}
          \centering
          \includegraphics[width=0.99\linewidth]{src/4/img/duplication.png}
          \caption{Success rates under the four fault models for the seven benchmark versions with software and/without duplication countermeasures}
          \label{hwsw combine}
        \end{figure}
    \end{column}
    \end{columns} 
\end{frame}
%%%%%%%%%%%%%%%%%%%%%%%%%%%%%%%%%%%%%%%%%%%%%%%%%%%%%%%%%%%%%%%%%%%%%%%%%%%%
\subsection{Proposed Triple-Redundant Countermeasure}
\begin{frame}{Proposed Architecture}
\begin{columns}[c]
  \column{0.5\textwidth}
  \begin{itemize}
    \item Purely hardware-based countermeasure for reliability and efficiency
    \item Detection-only strategy (higher accuracy, modest correction trade-off)
    \item Applied to \texttt{ack}, \texttt{sel}, and \texttt{grant} registers
  \end{itemize}

  \column{0.5\textwidth}
  \begin{figure}
    \centering
    \includegraphics[width=0.9\textwidth]{src/4/img/cm tri.png}
    \caption{Triplication-redundant detection scheme}
  \end{figure}
\end{columns}
\end{frame}

\begin{frame}{Proposed Countermeasure Results}
\begin{table}
  \centering
  \label{tab:our cm}
  \scalebox{0.5}{
\begin{tabular}{llccccc}
\hline
Countermeasure & Fault model & crash & detect\_hw & success & change & silence \\
\hline
& Bit-Flip & 86 & 4828 & 0 & 0 & 468 \\
& Manipulate Register & 125 & 5491 & 0 & 0 & 5148 \\
& 2 Bit-Flips & 2007 & 56961 & 0 & 0 & 234 \\
\multirow{-4}{*}{triplication-redundant v0} & Manipulate Two Registers & 5498 & 174916 & 0 & 0 & 53586 \\
\hline
& Bit-Flip & 109 & 6611 & 0 & 0 & 640 \\
& Manipulate Register & 161 & 7519 & 0 & 0 & 7040 \\
& 2 Bit-Flips & 2572 & 78068 & 0 & 0 & 320 \\
\multirow{-4}{*}{triplication-redundant v1} & Manipulate Two Registers & 7106 & 239614 & 0 & 0 & 73280 \\
\hline
& Bit-Flip & 170 & 10120 & 0 & 0 & 980 \\
& Manipulate Register & 242 & 11518 & 0 & 0 & 10780 \\
& 2 Bit-Flips & 3879 & 119601 & 0 & 0 & 490 \\
\multirow{-4}{*}{triplication-redundant v2} & Manipulate Two Registers & 10544 & 367246 & 0 & 0 & 112210 \\
\hline
& Bit-Flip & 102 & 7521 & 0 & 0 & 726 \\
& Manipulate Register & 150 & 8562 & 0 & 0 & 7986 \\
& 2 Bit-Flips & 2366 & 89110 & 0 & 0 & 363 \\
\multirow{-4}{*}{triplication-redundant v3} & Manipulate Two Registers & 6574 & 273299 & 0 & 0 & 83127 \\
\hline
& Bit-Flip & 174 & 10914 & 0 & 0 & 1056 \\
& Manipulate Register & 256 & 12416 & 0 & 0 & 11616 \\
& 2 Bit-Flips & 4046 & 129010 & 0 & 0 & 528 \\
\multirow{-4}{*}{triplication-redundant v4} & Manipulate Two Registers & 11228 & 395860 & 0 & 0 & 120912 \\
\hline
& Bit-Flip & 177 & 10386 & 0 & 0 & 1006 \\
& Manipulate Register & 249 & 11823 & 0 & 0 & 11066 \\
& 2 Bit-Flips & 3999 & 122757 & 0 & 0 & 503 \\
\multirow{-4}{*}{triplication-redundant v5} & Manipulate Two Registers & 11822 & 375991 & 0 & 0 & 115187 \\
\hline
& Bit-Flip & 115 & 8684 & 0 & 0 & 838 \\
& Manipulate Register & 169 & 9887 & 0 & 0 & 9218 \\
& 2 Bit-Flips & 2684 & 102904 & 0 & 0 & 419 \\
\multirow{-4}{*}{triplication-redundant v6} & Manipulate Two Registers & 7424 & 315625 & 0  & 0 & 95951 \\
\hline
& Bit-Flip & 200 & 13660 & 0 & 0 & 1320 \\
& Manipulate Register & 295 & 15545 & 0 & 0 & 14520 \\
& 2 Bit-Flips & 4683 & 161637 & 0 & 0 & 660 \\
\multirow{-4}{*}{triplication-redundant v7} & Manipulate Two Registers & 16041 & 492819 & 0 & 0 & 151140 \\         
\hline             
\end{tabular}
}
\end{table}
\end{frame}

%%%%%%%%%%%%%%%%%%%%%%%%%%%%%%%%%%%%%%%%%%%%%%%%%%%%%%%%%%%%%%%%%%%%%%%%%%%%
\begin{frame}{Resource Overhead of the Proposed Design}
\begin{itemize}
  \item LUT utilization increases by 0.36\%.
  \item Maximum operating frequency increases by 0.48\%.
  \item Changes attributed to Vivado synthesis/optimization heuristics.
  \item Overall practicality of the design remains unaffected.
\end{itemize}

\begin{table}
\centering
%\scriptsize
\caption{Triplication-redundant resource usage}
\begin{tabular}{lccc}
\hline
Countermeasure & LUT & Flip-Flops & Frequency (MHz) \\
\hline
Unprotected & 2198 & 1793 & 70.13 \\
Triplication-redundant & 2206 & 1791 & 70.47 \\
\hline
\end{tabular}
\end{table}
\end{frame}
%
% ---------------------------------------------------------------- %
\section{Conclusion}

\subsection{Synthesis}
\begin{frame}{Key Contributions}
    \begin{itemize}
        \item Identified vulnerabilities in SoC buses (Wishbone, AXI-Lite, AXI).
        \item Developed experiment framework using LiteX + VerifyPin + FISSA.
        \item Demonstrate limitation of software countermeasure
        \item Proposed hardware-based countermeasure for the Wishbone bus.
        \item Demonstrated full protection under evaluated fault models.
    \end{itemize}
\end{frame}

\begin{frame}{Experimental Highlights}
    \begin{itemize}
        \item Over 40 million simulations executed on 3 Xeon Gold servers. 
        \item Total campaign duration: more than 2 months.
    \end{itemize}
\end{frame}

%%%%%%%%%%%%%%%%%%%%%%%%%%%%%%%%%%%%%%%%%%%%%%%%%%%%%%%%%%%%%%%%%%%%%%%%%%%%
\subsection{Global Reflection}
\begin{frame}{Global Reflection}
    \begin{itemize}
        \item Bus-level vulnerabilities exist across Wishbone, AXI-Lite, AXI.
        \item Real-world SoCs lack built-in protection for handshake signals.
        \item Multi-bit fault injection remains a realistic threat with advanced equipment.
    \end{itemize}
\end{frame}

%%%%%%%%%%%%%%%%%%%%%%%%%%%%%%%%%%%%%%%%%%%%%%%%%%%%%%%%%%%%%%%%%%%%%%%%%%%%
\subsection{Perspectives}
\begin{frame}{Future Directions}
    \begin{itemize}
        \item Apply countermeasures to AXI-Lite and AXI buses.
        \item Combine advanced protections (CFI, runtime checks) against both software/hardware attack.
        \item Integrate protection into LiteX to add the security dimension.
        \item Evaluate NoC-based architectures under fault injection attacks.
    \end{itemize}
\end{frame}
%
% ---------------------------------------------------------------- %
\section*{Publications}

%%%%%%%%%%%%%%%%%%%%%%%%%%%%%%%%%%%%%%%%%%%%%%%%%%%%%%%%%%%%%%%%%%%%%%%%%%%%
\begin{frame}[allowframebreaks]{Publications}
    \begin{block}{International peer-reviewed conferences with proceedings}
    \begin{enumerate}
        \item {\footnotesize\textbf{Hongwei Zhao}, Vianney Lapotre, and Guy Gogniat. "Communication Architecture Under Siege: An In-depth Analysis of Fault Attack Vulnerabilities and Countermeasures."~\cite{zhao2024communication} \textit{2024 IEEE International Conference on Cyber Security and Resilience (CSR)}, IEEE, 2024. \textbf{Published}.}
        \item {\footnotesize\textbf{Hongwei Zhao}, Vianney Lapotre, and Guy Gogniat. "Fault Injection in On-Chip Interconnects: A Comparative Study of Wishbone, AXI-Lite, and AXI." \textit{Workshop on Design and Architectures for Signal and Image Processing}. \textbf{Minor revision}.}
        \item {\footnotesize\textbf{Hongwei Zhao}, Vianney Lapotre, and Guy Gogniat. "Analyzing and Mitigating Wishbone Bus Vulnerabilities in RISC-V SoC Architecture: A Hardware and Software Countermeasures Approach." \textit{IEEE Transactions on Dependable and Secure Computing}. \textbf{Rejected, resubmitted}.}
    \end{enumerate}
    \end{block}
\end{frame}
%%%%%%%%%%%%%%%%%%%%%%%%%%%%%%%%%%%%%%%%%%%%%%%%%%%%%%%%%%%%%%%%%%%%%%%%%%%%

\begin{frame}
    \vfill
    \centering
    \begin{beamercolorbox}[sep=8pt,center,shadow=true,rounded=true]{title}
        \usebeamerfont{section title} Thank you for attention!
    \end{beamercolorbox}
    \vfill
\end{frame}
%
% ---------------------------------------------------------------- %

\begin{frame}[noframenumbering]
    \vfill
    \centering
    \begin{beamercolorbox}[sep=8pt,center,shadow=true,rounded=true]{title}
        \usebeamerfont{section title} References
    \end{beamercolorbox}
    \vfill
\end{frame}

\begin{frame}[allowframebreaks, noframenumbering]{References}
    \printbibliography
    \nocite{flaticon}
\end{frame}
%
% ---------------------------------------------------------------- %
\section*{Back up}

\begin{frame}{Communication architecture under fault injection}
        \begin{block}{}
            \begin{itemize}
                \item Exist article talk about EMP injection on the bus, point out the risk of bus~\cite{mishra2024faults}
                \item More global research is needed 
            \end{itemize}
        \end{block}
\end{frame}

\begin{frame}{Why a Custom SoC?}
    \begin{block}{Motivation}
        \begin{itemize}
            \item Vendor-locked SoCs lack transparency and configurability.
            \item Need precise control over interconnects, memory, and processor interfaces.
            \item Enable reproducible fault injection experiments.
        \end{itemize}
    \end{block}
\end{frame}

\begin{frame}{FISSA Workflow}
    \begin{enumerate}
        \item Parse configuration (JSON) and targets (YAML).
        \item Generate TCL scripts for fault injection.
        \item Run simulations and collect logs.
    \end{enumerate}
\end{frame}

\begin{frame}{Configuration file}
    \begin{columns}
        \begin{column}{0.5\textwidth}
        \begin{itemize}
            \item path\_...: Defines simulator and all required file paths.
            \item threat\_model: Selects the fault model (here: spatial bit-flip)
            \item avoid\_register \& avoid\_log\_registers: Optional exclusion/log lists for registers (unused in our setup)
            \item target\_window: Sets the fault-injection window based on VerifyPIN execution cycles
            \item cycle\_ref: Specifies total cycles to observe the final authentication outcome
            \item cpu\_period \& batch\_sim: Uses an 8 ns CPU period and groups 4000 simulations per batch
        \end{itemize}
        \end{column}
    \begin{column}{0.5\textwidth}
        \begin{figure}
        \centering
        \includegraphics[width=0.9\textwidth]{src/2/img/config.png}
        \caption{config.json file}
        \end{figure}
    \end{column}
    \end{columns}
\end{frame}

\begin{frame}{Register file}
    \begin{columns}
        \begin{column}{0.4\textwidth}
        \begin{itemize}
            \item Each entry lists a full hierarchical register name.
            \item A bit-width value specifies how many bits are injected.
            \item Example: the signal \texttt{builder\_axirequestcounter0\_full} shown is configured as a 1-bit injection target.
        \end{itemize}
        \end{column}
    \begin{column}{0.6\textwidth}
        \begin{figure}
        \centering
        \includegraphics[width=0.9\textwidth]{src/2/img/yaml.png}
        \caption{yaml file}
        \end{figure}
    \end{column}
    \end{columns}
\end{frame}

\begin{frame}{Global Vue}
    \begin{itemize}
        \item Table~\ref{Vunerabilities on 3 buses} compares outcomes for all fault models across Wishbone, AXI-Lite, and AXI.
        \item More simulations are required as bus complexity increases (Wishbone → AXI-Lite → AXI).
        \item Under simple models, Wishbone exhibits higher attack success, showing that simpler interconnects are easier to disrupt.
        \item Subsequent analysis examines each successful register-fault combination in detail.
    \end{itemize}
\end{frame}

\begin{frame}{Precise analysis on AXI}
    \begin{columns}
        \begin{column}{0.4\textwidth}
            \begin{itemize}
                \item State register is the main target for successful attacks across all fault models.
                \item completion flag register shows fewer successful attacks.
            \end{itemize}
        \end{column}
    \begin{column}{0.6\textwidth}
        \begin{table}
        \centering
        \caption{Distribution of successful register combination attacks under different fault models on the AXI Bus}
        \label{Percentage on AXI}
        \begin{tabularx}{\textwidth}{|l|X|X|X|}
        \hline
        Fault model & state & completion flag & completion flag \& state \\
        \hline
        Bit-flip & 75.00\% & 25\% & - \\
        Manipulate Register & 83.33\% & 16.67\% & - \\
        2 Bit-Flips & 60.00\% & 20\% & 20.00\% \\
        Manipulate Two Registers & 83.13\% & 16.79\% & 0.08\% \\
        \hline
        \end{tabularx}
        \end{table}
    \end{column}
    \end{columns}
\end{frame}

\begin{frame}{Fault Effect Types}
\begin{itemize}
    \item Instruction skip: Faults (e.g., in ack) disrupt fetch timing, causing key instructions—such as the PIN comparison—to be skipped.
    \item Data reset: Faults (e.g., in state) trigger error-handling behavior, resetting bus outputs to zero and making g\_cardPin appear equal to g\_userPin.
    \item Data misread: Address-selection faults (e.g., in sel) cause reads from unintended modules, substituting variables like g\_userPin with g\_cardPin.
    \item Data multiread: Faulty sel may enable multiple memories at once; merged outputs (e.g., ORed values) corrupt data used in authentication.
\end{itemize}
\end{frame}

\begin{frame}{SEL Under Attack}
    \begin{columns}
        \begin{column}{0.4\textwidth}
        \begin{itemize}
          \item \textbf{Data reset:} Fault forces \texttt{sel} = "0000", masking SRAM data with zeros.
          \item \textbf{Data misread:} Single-bit error in \texttt{sel} causes ROM to be read instead of SRAM.
          \item \textbf{Data multiread:} Multiple unintended bits set in \texttt{sel} (e.g., "1100"), CPU reads CSR + MAIN\_RAM simultaneously.
        \end{itemize}
        \end{column}
    \begin{column}{0.6\textwidth}
        \begin{figure}
            \centering
            \includegraphics[width=0.7\textwidth]{src/4/img/fault3.png}
            \caption{Impact of sel signal on address, data signals and CPU cache without an attack}
        \end{figure}
    \end{column}
    \end{columns} 
\end{frame}

\begin{frame}{Software Countermeasures Overview}
\begin{itemize}
  \item V7 showed the lowest fault injection success rate and highest detection capability.
  \item Countermeasure complexity (code length, CCN, token count) correlated with longer execution time and larger fault surface.
  \item Misaligned countermeasures (e.g., V4--V5) produced higher success rates than baseline, failing to intercept relevant fault paths.
  \item None of the countermeasures fully neutralized the fault model impact.
  \item Findings suggest need for granular analysis of countermeasure logic and deployment.
\end{itemize}
\end{frame}

\begin{frame}{Grant and ACK Under Attack}
    \begin{columns}
        \begin{column}{0.4\textwidth}
            \begin{itemize}
              \item \texttt{grant} from 0 to 1, CPU begins to read from SRAM, ack1 becomes 1
              \item \texttt{grant} becomes 1 one period before, so as ack1
              \item Combine attack with ack0, cause ack\_d =1 during 2 periods, influence 2 data only during 1 period on the bus, cause the same fault as before.
            \end{itemize}
        \end{column}
    \begin{column}{0.6\textwidth}
        \begin{figure}
            \centering
            \includegraphics[width=0.8\textwidth]{src/4/img/fault2.png}
            \caption{Impact of grant and acknowledge signal on address, data signals and CPU cache without an attack}
        \end{figure}
    \end{column}
    \end{columns} 
\end{frame}

\begin{frame}{Software Countermeasures Overview}
\begin{itemize}
  \item V0 / V1: HB reduce little instruction fault.
  \item V1 / V4: FTL and LC reduce less instruction fault than INL increases, LC reduce many data fault.
  \item V4 / V5: Double-call reduce instruction fault and data fault.
\end{itemize}

\begin{table}
  \centering
  \label{tab:fault analysis}
\small
\begin{tabular}{llrr}
\hline
Benchmark version & CM implemented & \multicolumn{1}{l}{Instruction success times} & \multicolumn{1}{l}{Data success times} \\
\hline
V0 & - & 20 & 17 \\
V1 & HB & 19 & 17 \\
V4 & HB+FTL+INL+DPTC+PTCBK+LC & 26 & 5 \\
V5 & HB+FTL+DPTC+DC & 4 & 5 \\
\hline
\end{tabular}
\end{table}
\end{frame}
%
% ---------------------------------------------------------------- %
\end{document}
