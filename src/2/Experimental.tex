\section{Experimental Setup}

%%%%%%%%%%%%%%%%%%%%%%%%%%%%%%%%%%%%%%%%%%%%%%%%%%%%%%%%%%%%%%%%%%%%%%%%%%%%
\subsection{Construction of the SoC}

\begin{frame}{LiteX Framework}
    \begin{columns}
        \begin{column}{0.45\textwidth}
        \begin{itemize}
            \item Open-source SoC builder for FPGA-based systems \cite{litex}.
            \item Modular design with Python-based HDL (Migen).
            \item Supports several processor architectures (e.g. RISC-V).
            \item No built-in security → interesting to explore vulnerability analysis in this functional-oriented architecture.
        \end{itemize}
    \end{column}
    \begin{column}{0.55\textwidth}
    \begin{figure}
        \centering
        \includegraphics[width=0.7\textwidth]{src/2/img/litex.png}
        \caption{LiteX framework components}
    \end{figure}    
    \end{column}
    \end{columns}
\end{frame}

%%%%%%%%%%%%%%%%%%%%%%%%%%%%%%%%%%%%%%%%%%%%%%%%%%%%%%%%%%%%%%%%%%%%%%%%%%%%
\subsection{SoC Configuration}
\begin{frame}{SoC Architecture Overview}
    \begin{itemize}
        \item CPU: VexRiscv (RISC-V ISA).
        \item Interconnect protocols: Wishbone, AXI-Lite, AXI.
        \item Memory regions: ROM, SRAM, CSR, MAIN\_RAM.
        \item Target FPGA: Digilent Basys3 (Artix-7).
    \end{itemize}
\begin{figure}
    \centering
    \includegraphics[width=0.4\textwidth]{src/2/img/socgeneral.png}
    \caption{Our SoC architecture}
\end{figure}
\end{frame}

%%%%%%%%%%%%%%%%%%%%%%%%%%%%%%%%%%%%%%%%%%%%%%%%%%%%%%%%%%%%%%%%%%%%%%%%%%%%
\subsection{Benchmark}
\begin{frame}{VerifyPin Benchmark}
    \begin{itemize}
        \item Written in C \cite{DureuilPPLCC16}.
        \item Simulates a PIN verification process.
        \item Suite of 8 implementations: V0 (unprotected) + V1–V7 (protected).
    \end{itemize}
\end{frame}

\begin{frame}{VerifyPin V0 Example}
    \begin{columns}
        \begin{column}{0.5\textwidth}
        \begin{itemize}
            \item Compares user PIN (\texttt{"0000"}) with card PIN (\texttt{"4321"}).
            \item Fault success: \texttt{g\_authenticated} set to 1 despite mismatch.
        \end{itemize}
        \end{column}
    \begin{column}{0.5\textwidth}
        \begin{figure}
        \centering
        \includegraphics[width=0.9\textwidth]{src/2/img/v0.png}
        \caption{C code of VerifyPin function in benchmark V0}
        \end{figure}
    \end{column}
    \end{columns}
\end{frame}

\begin{frame}{Countermeasures implemented in V1 to V7}
\begin{itemize}
  \item HB: Hardened Boolean
  \item FTL: Fixed-Time Loop
  \item INL: Inlined Function
  \item DPTC/PTCBK: Token Counter decremented first/Back up
  \item LC: Loop Counter
  \item DC/DT: Double Call/Test
  \item SC: Step Counter
\end{itemize}
\end{frame}

\begin{frame}{VerifyPin V1 Example}
    \begin{columns}
        \begin{column}{0.5\textwidth}
        \begin{itemize}
            \item Implement with Hardened Boolean.
            \item Replace 1 and 0 with BOOL\_True(0xAA) and BOOL\_False(0x55).
            \item Detect fault => execute countermeasure()
        \end{itemize}
        \end{column}
    \begin{column}{0.5\textwidth}
        \begin{figure}
        \centering
        \includegraphics[width=0.9\textwidth]{src/2/img/v1.png}
        \caption{C code of VerifyPin function in benchmark V1}
        \end{figure}
    \end{column}
    \end{columns}
\end{frame}
%%%%%%%%%%%%%%%%%%%%%%%%%%%%%%%%%%%%%%%%%%%%%%%%%%%%%%%%%%%%%%%%%%%%%%%%%%%%
\subsection{FISSA Tool}
\begin{frame}{FISSA Overview}
    \begin{itemize}
        \item Python-based tool for fault injection campaigns \cite{fissa}.
        \item Works with HDL simulators (Questasim, Vivado, Verilator).
        \item Automates TCL script generation and simulation logging.
    \end{itemize}
\begin{figure}
    \centering
    \includegraphics[width=0.8\textwidth]{src/2/img/fissa.png}
    \caption{FISSA components}
\end{figure}
\end{frame}

\begin{frame}{Register lists}
    \begin{itemize}
        \item Non-control registers (I/O, LEDs, clock/reset, timers, program data) were excluded due to low injection effectiveness.
        \item Fault injection was applied to Wishbone, AXI-Lite, and AXI under multiple fault models.
    \end{itemize}
        
    \begin{table}
    \centering
    \caption{All the registers targeted by fault injection in the 3 buses}
    \label{inject reg}
    \resizebox{\textwidth}{!}{
    \begin{tabular}{|l|l|}
    \hline
    Bus & Registers \\
    \hline
    Wishbone & ACK, SEL, done, grant \\
    AXI-Lite & state, selection driver, last\_was\_read, rr\_read\_grant, completion flags \\
    AXI & same with AXI-Lite, ax\_beat\_first, ax\_beat\_last, last\_ar\_aw\_n, pipe\_valid\_source \\
    \hline
    \end{tabular}}
    \end{table}
\end{frame}

\begin{frame}{Considered Fault Models}
    \begin{enumerate}
        \item Bit-Flip: single bit in register.
        \item Manipulate Register: arbitrary bit-flips in one register.
        \item 2 Bit-Flips: exactly two bits flipped.
        \item Manipulate Two Registers: arbitrary bit-flips in two registers.
    \end{enumerate}
\begin{figure}
    \centering
    \includegraphics[width=0.8\textwidth]{src/2/img/fault model.png}
    \caption{Illustration of our fault models}
\end{figure}
\end{frame}

\begin{frame}{Outcome Categories}
    \begin{itemize}
        \item Crash: simulation terminated.
        \item Software detection: software countermeasure to detect the fault.
        \item Hardware detection: hardware countermeasure to detect the fault.
        \item Hardware correction: hardware countermeasure to correct the fault.
        \item Success: unauthorized authentication achieved.
        \item Change: memory state altered.
        \item Silence: no visible impact.
    \end{itemize}
\end{frame}
%%%%%%%%%%%%%%%%%%%%%%%%%%%%%%%%%%%%%%%%%%%%%%%%%%%%%%%%%%%%%%%%%%%%%%%%%%%%
\subsection{Fault Models}
\begin{frame}{Fault Injection Campaign}
    \begin{itemize}
        \item Attacker has no PIN knowledge.
        \item Attacker has knowledge of the bus architecture.
        \item Faults injected at the RTL level via simulation.
        \item Target: control registers connected to the system bus.
    \end{itemize}
\end{frame}