\section{Protection Implementation on Wishbone Bus}

\subsection{Wishbone Vulnerabilities}
\begin{frame}{Critical Signals Under Attack}
    \begin{itemize}
        \item \texttt{ack}: 4-bit register, controlled by sel register, cyc/stb signal, assign to ack\_d and ack\_i
        \item \texttt{sel}: 4-bit register, assigned by adr register
        \item \texttt{grant}: arbitration register for ack\_d/ack\_i, etc
        
    \end{itemize}

    \begin{columns}
        \begin{column}{0.55\textwidth}
        \begin{figure}
            \centering
            \includegraphics[width=0.7\textwidth]{src/3/img/wishbone bus.png}
        \end{figure}
        \end{column}
    \begin{column}{0.45\textwidth}
        \begin{table}
        \centering
        \caption{Distribution of successful register combination attacks under different fault models on the Wishbone Bus}
        \label{Percentage on Wishbone}
          \resizebox{\columnwidth}{!}{
        \begin{tabular}{lcccc}
        \hline
        Fault model & \texttt{ack} & \texttt{sel} & \texttt{ack} \& \texttt{grant} & \texttt{ack} \& \texttt{sel} \\ 
        \hline
        Bit-flip & 94.59\% & 5.41\% & - & - \\
        Manipulate Register & 81.40\% & 18.60\% & - & - \\
        2 Bit-Flips & 94.12\% & 3.68\% & 1.10\% & 1.10\% \\
        Manipulate Two Registers & 85.73\% & 12.34\% & 0.38\% & 1.55\% \\
        \hline
        \end{tabular}
        }
        \end{table}
    \end{column}
    \end{columns} 
    
\end{frame}

\begin{frame}{ACK Under Attack}
    \begin{columns}
        \begin{column}{0.4\textwidth}
          \textbf{Normal behavior:}
            \begin{itemize}
              \item \texttt{ack1} toggles each cycle while other \texttt{ack} signals remain low.
              \item \texttt{ack\_d} updates both address and data every two cycles.
              \item CPU sequentially reads values and commits them to cache.
            \end{itemize}
        \end{column}
    \begin{column}{0.6\textwidth}
        \begin{figure}
            \centering
            \includegraphics[width=0.9\textwidth]{src/4/img/nofault1.png}
            \caption{Impact of acknowledge signal on address, data signals and CPU cache without an attack}
        \end{figure}
    \end{column}
    \end{columns} 
\end{frame}

\begin{frame}{ACK Under Attack}
    \begin{columns}
        \begin{column}{0.4\textwidth}
          \textbf{Faulted behavior:}
            \begin{itemize}
              \item A bit-flip on \texttt{ack0} disrupts the expected read timing.
              \item Two-cycle transfers collapse into a single cycle, causing premature cache allocation.
              \item Cache lines are filled with corrupted values (e.g., PIN replaced by zeros).
              \item The program compares identical PINs and incorrectly grants access.
              \item It corresponds to an Instruction skip
            \end{itemize}
        \end{column}
        \begin{column}{0.6\textwidth}
            \begin{figure}
                \centering
                \includegraphics[width=0.9\textwidth]{src/4/img/fault1.png}
                \caption{Impact of acknowledge signal on address, data signals and CPU cache with an attack}
            \end{figure}
        \end{column}
    \end{columns} 
\end{frame}


%%%%%%%%%%%%%%%%%%%%%%%%%%%%%%%%%%%%%%%%%%%%%%%%%%%%%%%%%%%%%%%%%%%%%%%%%%%%
\subsection{Software Countermeasures}

\begin{frame}{Countermeasure Deployment}
\begin{table}
\centering
  \caption{Software countermeasures deployed in each benchmark version}
  \label{tab:benchmark cm}
  \begin{tabular}{cccccccccc}
    \hline
    & HB & FTL & INL & DPTC & PTCBK & LC & DC & DT & SC\\
    \hline
    \texttt{V0} & & & & & & & & &  \\
    \texttt{V1} & \checkmark & & & & & & & &  \\
    \texttt{V2} & \checkmark & \checkmark & & & & & & & \\
    \texttt{V3} & \checkmark & \checkmark & \checkmark & & & & & & \\
    \texttt{V4} & \checkmark & \checkmark & \checkmark & \checkmark & \checkmark & \checkmark & & &\\
    \texttt{V5} & \checkmark & \checkmark & & \checkmark & & & \checkmark & &  \\
    \texttt{V6} & \checkmark & \checkmark & \checkmark & \checkmark & & & & \checkmark &  \\
    \texttt{V7} & \checkmark & \checkmark & \checkmark & \checkmark & & & & \checkmark & \checkmark\\
    \hline
  \end{tabular}
\end{table}
\end{frame}

%%%%%%%%%%%%%%%%%%%%%%%%%%%%%%%%%%%%%%%%%%%%%%%%%%%%%%%%%%%%%%%%%%%%%%%%%%%%
\subsection{Software Countermeasure Results}
\begin{frame}{Fault Injection Results et Resource Analysis Lizard (Software)}
\begin{itemize}
  \item 8 versions of the benchmark with/without software countermeasure
  \item Fault model: Bitflip
  \item Lizard: analysis NLOC, CCN, token
\end{itemize}
\begin{table}
\captionsetup{skip=10pt}
  \centering
  \label{tab:benchmark result}
  \resizebox{\textwidth}{!}{%
\begin{tabular}{cccccccccccc}
\hline
& crash & detect & success & change & silence & success rate & detect rate & sum & NLOC & CCN & token \\
\hline
V0 & 47 & 0 & 37 & 124 & 2366 & 1.44\% & 0 & 2574 & 32 & 6 & 107 \\
V1 & 61 & 21 & 36 & 136 & 3266 & 1.02\% & 0.60\% & 3520 & 36 & 7 & 127 \\
V2 & 98 & 20 & 30 & 164 & 5078 & 0.56\% & 0.37\% & 5390 & 44 & 8 & 149 \\
V3 & 58 & 19 & 31 & 96 & 3789 & 0.78\% & 0.48\% & 3993 & 32 & 7 & 129 \\
V4 & 92 & 101 & 33 & 104 & 5478 & 0.57\% & 1.74\% & 5808 & 47 & 11 & 191 \\
V5 & 105 & 28 & 9 & 153 & 5238 & 0.16\% & 0.51\% & 5533 & 42 & 9 & 163 \\
V6 & 61 & 49 & 14 & 108 & 4377 & 0.30\% & 1.06\% & 4609 & 38 & 9 & 153 \\
V7 & 105 & 186 & 9 & 252 & 6708 & 0.12\% & 2.56\% & 7260 & 77 & 18 & 312\\       
\hline                          
\end{tabular}
}
\end{table}
\end{frame}

\begin{frame}{Software Countermeasures Overview}
  Instruction protection:
\begin{itemize}
  \item HB: Prevents register values defaulting to 0/1, mitigating some branch instruction.
  \item FTL/LC: Fixes/Records  loop iteration count, blocking loop manipulation.
  \item INL: Merges functions can't reduce fault.
  \item DPTC/PTCBK: State counters not targeted, no effective defense observed.
  \item DC/DT: Redundant execution neutralizes function-call attacks.
  \item SC: Detects skipped instructions, effective against instruction-skipping.
\end{itemize}
  Data protection: Only INL (fewer reads) and DC (double read) provide defense.
\begin{table}
  \caption{Evaluation of software countermeasures against fault attacks on instructions and data}
  \label{tab:cm evaluate}
  \resizebox{\columnwidth}{!}{
\begin{tabular}{lccccccccc}
\hline
& HB & FTL & INL & DPTC & PTCBK & LC & DC & DT & SC \\
\hline
Instruction fault & \cmark & \cmark & \xmark & \xmark & \xmark & \cmark & \cmark & \cmark & \cmark \\
Data fault        & \xmark & \xmark & \cmark & \xmark & \xmark & \xmark & \cmark & \xmark & \xmark\\
\hline
\end{tabular}
}
\end{table}
\end{frame}

%%%%%%%%%%%%%%%%%%%%%%%%%%%%%%%%%%%%%%%%%%%%%%%%%%%%%%%%%%%%%%%%%%%%%%%%%%%%
\subsection{Hardware Countermeasures}
\begin{frame}{Hardware Countermeasures Overview}
\begin{columns}
  \column{0.5\textwidth}
  \begin{itemize}
    \item Software-only defenses are limited, Hardware countermeasures are needed.
    \item Architectural change: combi=> mux, reducing multiple-read attacks.
    \item Hardware protections on \texttt{ack} and \texttt{sel} registers.
  \end{itemize}

  \column{0.5\textwidth}
  \begin{figure}
    \centering
    \includegraphics[width=\textwidth]{src/4/img/change arch.png}
    \caption{Optimized connection between the selection register and the memory using multiplexers}
  \end{figure}
\end{columns}
\end{frame}

\begin{frame}{Hardware Countermeasures Overview}
\begin{itemize}
\item Simple Parity: Detects faults using a 1-bit parity code.
\item Duplication: Creates a duplicate of the registers and compares it with the unprotected version. 
\item Complementary Duplication: Duplicates the inverse of the registers and compares it with the unprotected version.
\item Hamming Code: Corrects the output signal of a register using a 3-bit checksum.
\item SECDED Code: Corrects or detects errors using a 4-bit checksum.
\item Triplication: Duplicates a register twice, correcting the signal if two registers have matching outputs and detecting errors if all three differ.
\end{itemize}
\end{frame}

%%%%%%%%%%%%%%%%%%%%%%%%%%%%%%%%%%%%%%%%%%%%%%%%%%%%%%%%%%%%%%%%%%%%%%%%%%%%
\begin{frame}{Results of Hardware Countermeasure with VerifyPin V0}
\begin{table}
\captionsetup{skip=10pt}
\centering
  \label{tab:v0 under attack result}
\scalebox{0.6}{
\begin{tabular}{llccc}
\hline
Countermeasure & Fault model & Success rate & Detection rate & Correction rate \\
\hline
 & Bit-Flip & 1.44\% & - & - \\
 & Manipulate Register & 0.61\% & - & - \\
 & 2 Bit-Flips & 2.11\% & - & - \\
\multirow{-4}{*}{Unprotected} & Manipulate Two Registers  & 1.08\%  & - & - \\
\hline
 & Bit-Flip & 0\% & 69.54\% & - \\
 & Manipulate Register & 0.04\% & 34.77\% & - \\
 & 2 Bit-Flips & 0.89\% & 64.38\% & - \\
\multirow{-4}{*}{Simple parity} & Manipulate Two Registers & 0.27\% & 50.34\% & - \\
\hline
 & Bit-Flip & 0\% & 77.66\% & - \\
 & Manipulate Register & 0\% & 44.94\% & - \\
 & 2 Bit-Flips & 0.15\% & 86.43\% & - \\
\multirow{-4}{*}{Duplication} & Manipulate Two Registers & 0.04\% & 66.46\% & - \\
\hline
 & Bit-Flip & 0\% & 77.66\% & - \\
Complementary & Manipulate Register & 0\% & 44.94\% & - \\
Duplication & 2 Bit-Flips & 0.15\% & 86.43\% & - \\
 & Manipulate Two Registers & 0.04\% & 66.46\% & - \\
\hline
 & Bit-Flip & 0\% & - & 80\% \\
 & Manipulate Register & 0.49\% & - & 58.16\% \\
 & 2 Bit-Flips & 0.97\% & - & 91.30\% \\
\multirow{-4}{*}{Hamming code} & Manipulate Two Registers & 1.01\% & - & 75.95\% \\
\hline
 & Bit-Flip & 0\% & 0\% & 85.71\% \\
 & Manipulate Register & 0\% & 0\% & 81.82\% \\
 & 2 Bit-Flips & 0.23\% & 20\% & 76.94\% \\
\multirow{-4}{*}{Triplication} & Manipulate Two Registers & 0.14\% & 36.74\% & 58.97\% \\
\hline
 & Bit-Flip & 0\% & 11.77\% & 70.59\% \\
 & Manipulate Register & 0.32\% & 34\% & 37.92\% \\
 & 2 Bit-Flips & 0\% & 45.56\% & 52.06\% \\
\multirow{-4}{*}{Secded code} & Manipulate Two Registers & 0.45\% & 51.94\% & 36.19\% \\
\hline
\end{tabular}
}
\end{table}
\end{frame}

% \begin{frame}{Hardware Countermeasures Overview}
% \begin{itemize}
%   \item Duplication / complementary duplication: lower success rates, higher detection probability.
%   \item SECDED (Hamming code): strong correction, effective against dual-bit faults across registers.
%   \item Triplication: best for single-bit errors and localized multi-bit corruption.
%   \item All methods defend against single bit-flip.
%   \item Duplication, complementary duplication, triplication: effective for multi-bit faults in one register.
%   \item SECDED: defends against two bit-flips.
%   \item No method fully prevents multi-register multi-bit faults; triplication performs better.
%   \item Vulnerabilities:
%     \begin{itemize}
%       \item Simple parity: weak against two bit-flips (parity + signal line).
%       \item SECDED: vulnerable to multi-register manipulations, possible false corrections.
%       \item Duplication / complementary duplication / triplication: susceptible to two bit-flips mirroring errors.
%     \end{itemize}
% \end{itemize}
% \end{frame}

%%%%%%%%%%%%%%%%%%%%%%%%%%%%%%%%%%%%%%%%%%%%%%%%%%%%%%%%%%%%%%%%%%%%%%%%%%%%
\subsection{Hardware Resource Overhead}
\begin{frame}{Resource Overhead Analysis}
\begin{itemize}
  \item Countermeasures increase LUT usage by max. 0.7\%.
  \item Frequency reduced by up to 0.97\%.
  \item Differences largely due to synthesizer auto-optimization.
  \item Overall: negligible additional hardware resource loss.
\end{itemize}

\begin{table}
    \centering
  \caption{Hardware resource overhead of each hardware countermeasure (LUT, Flip-Flop, frequency)}
  \label{tab:countermeasures synthesis}
\begin{tabular}{cccc}
\hline
Countermeasure & LUT & Flip-Flops & frequency (MHz) \\
\hline
Unprotected & 2198 & 1793 & 70.13 \\
Simple parity & 2214 & 1791 & 70.27 \\
Duplication & 2201 & 1791 & 70.18 \\
Complimentary & 2199 & 1791 & 70.37 \\
Hamming code & 2199 & 1794 & 70.32 \\
Triplication & 2199 & 1791 & 70.27 \\
Secded code & 2193 & 1789 & 69.44 \\
\hline
\end{tabular}
\end{table}
\end{frame}

%%%%%%%%%%%%%%%%%%%%%%%%%%%%%%%%%%%%%%%%%%%%%%%%%%%%%%%%%%%%%%%%%%%%%%%%%%%%
\subsection{Compare Countermeasures}
\begin{frame}{Comparison of the Protection Effectiveness}
\begin{itemize}
  \item Hardware-only protections achieve consistently lower attack success rates.
  \item In some fault models, hardware countermeasures reduce success rate to zero.
\end{itemize}

    \begin{columns}
        \begin{column}{0.5\textwidth}
        \begin{figure}
          \centering
          \includegraphics[width=0.7\linewidth]{src/4/img/hw rate.png}
          \caption{Success rates under the four fault models for the benchmark V0 with different hardware countermeasures}
          \label{hw cmp}
        \end{figure}
        \end{column}
    \begin{column}{0.5\textwidth}
        \begin{figure}[H]
          \centering
          \includegraphics[width=0.7\linewidth]{src/4/img/sw rate.png}
          \caption{Success rates under the four fault models for the seven benchmark versions with software countermeasures}
          \label{sw cmp}
        \end{figure}
    \end{column}
    \end{columns} 
\end{frame}

\begin{frame}{Comparison of the Protection Effectiveness}
    \begin{columns}
        \begin{column}{0.5\textwidth}
        \begin{itemize}
          \item Figure~\ref{hwsw combine}: clear improvement with duplication + software vs. duplication alone.
          \item No hardware/software combination fully neutralizes the Manipulate Two Registers fault model.
          \item Persistent vulnerability indicates need for more advanced or hybridized protection beyond duplication/redundancy.
        \end{itemize}
        \end{column}
    \begin{column}{0.5\textwidth}
        \begin{figure}
          \centering
          \includegraphics[width=0.99\linewidth]{src/4/img/duplication.png}
          \caption{Success rates under the four fault models for the seven benchmark versions with software and/without duplication countermeasures}
          \label{hwsw combine}
        \end{figure}
    \end{column}
    \end{columns} 
\end{frame}
%%%%%%%%%%%%%%%%%%%%%%%%%%%%%%%%%%%%%%%%%%%%%%%%%%%%%%%%%%%%%%%%%%%%%%%%%%%%
\subsection{Proposed Triple-Redundant Countermeasure}
\begin{frame}{Proposed Architecture}
\begin{columns}[c]
  \column{0.5\textwidth}
  \begin{itemize}
    \item Purely hardware-based countermeasure for reliability and efficiency
    \item Detection-only strategy (higher accuracy, modest correction trade-off)
    \item Applied to \texttt{ack}, \texttt{sel}, and \texttt{grant} registers
  \end{itemize}

  \column{0.5\textwidth}
  \begin{figure}
    \centering
    \includegraphics[width=0.9\textwidth]{src/4/img/cm tri.png}
    \caption{Triplication-redundant detection scheme}
  \end{figure}
\end{columns}
\end{frame}

\begin{frame}{Proposed Countermeasure Results}
\begin{table}
  \centering
  \label{tab:our cm}
  \scalebox{0.5}{
\begin{tabular}{llccccc}
\hline
Countermeasure & Fault model & crash & detect\_hw & success & change & silence \\
\hline
& Bit-Flip & 86 & 4828 & 0 & 0 & 468 \\
& Manipulate Register & 125 & 5491 & 0 & 0 & 5148 \\
& 2 Bit-Flips & 2007 & 56961 & 0 & 0 & 234 \\
\multirow{-4}{*}{triplication-redundant v0} & Manipulate Two Registers & 5498 & 174916 & 0 & 0 & 53586 \\
\hline
& Bit-Flip & 109 & 6611 & 0 & 0 & 640 \\
& Manipulate Register & 161 & 7519 & 0 & 0 & 7040 \\
& 2 Bit-Flips & 2572 & 78068 & 0 & 0 & 320 \\
\multirow{-4}{*}{triplication-redundant v1} & Manipulate Two Registers & 7106 & 239614 & 0 & 0 & 73280 \\
\hline
& Bit-Flip & 170 & 10120 & 0 & 0 & 980 \\
& Manipulate Register & 242 & 11518 & 0 & 0 & 10780 \\
& 2 Bit-Flips & 3879 & 119601 & 0 & 0 & 490 \\
\multirow{-4}{*}{triplication-redundant v2} & Manipulate Two Registers & 10544 & 367246 & 0 & 0 & 112210 \\
\hline
& Bit-Flip & 102 & 7521 & 0 & 0 & 726 \\
& Manipulate Register & 150 & 8562 & 0 & 0 & 7986 \\
& 2 Bit-Flips & 2366 & 89110 & 0 & 0 & 363 \\
\multirow{-4}{*}{triplication-redundant v3} & Manipulate Two Registers & 6574 & 273299 & 0 & 0 & 83127 \\
\hline
& Bit-Flip & 174 & 10914 & 0 & 0 & 1056 \\
& Manipulate Register & 256 & 12416 & 0 & 0 & 11616 \\
& 2 Bit-Flips & 4046 & 129010 & 0 & 0 & 528 \\
\multirow{-4}{*}{triplication-redundant v4} & Manipulate Two Registers & 11228 & 395860 & 0 & 0 & 120912 \\
\hline
& Bit-Flip & 177 & 10386 & 0 & 0 & 1006 \\
& Manipulate Register & 249 & 11823 & 0 & 0 & 11066 \\
& 2 Bit-Flips & 3999 & 122757 & 0 & 0 & 503 \\
\multirow{-4}{*}{triplication-redundant v5} & Manipulate Two Registers & 11822 & 375991 & 0 & 0 & 115187 \\
\hline
& Bit-Flip & 115 & 8684 & 0 & 0 & 838 \\
& Manipulate Register & 169 & 9887 & 0 & 0 & 9218 \\
& 2 Bit-Flips & 2684 & 102904 & 0 & 0 & 419 \\
\multirow{-4}{*}{triplication-redundant v6} & Manipulate Two Registers & 7424 & 315625 & 0  & 0 & 95951 \\
\hline
& Bit-Flip & 200 & 13660 & 0 & 0 & 1320 \\
& Manipulate Register & 295 & 15545 & 0 & 0 & 14520 \\
& 2 Bit-Flips & 4683 & 161637 & 0 & 0 & 660 \\
\multirow{-4}{*}{triplication-redundant v7} & Manipulate Two Registers & 16041 & 492819 & 0 & 0 & 151140 \\         
\hline             
\end{tabular}
}
\end{table}
\end{frame}

%%%%%%%%%%%%%%%%%%%%%%%%%%%%%%%%%%%%%%%%%%%%%%%%%%%%%%%%%%%%%%%%%%%%%%%%%%%%
\begin{frame}{Resource Overhead of the Proposed Design}
\begin{itemize}
  \item LUT utilization increases by 0.36\%.
  \item Maximum operating frequency increases by 0.48\%.
  \item Changes attributed to Vivado synthesis/optimization heuristics.
  \item Overall practicality of the design remains unaffected.
\end{itemize}

\begin{table}
\centering
%\scriptsize
\caption{Triplication-redundant resource usage}
\begin{tabular}{lccc}
\hline
Countermeasure & LUT & Flip-Flops & Frequency (MHz) \\
\hline
Unprotected & 2198 & 1793 & 70.13 \\
Triplication-redundant & 2206 & 1791 & 70.47 \\
\hline
\end{tabular}
\end{table}
\end{frame}