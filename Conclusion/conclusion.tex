\chapter{Conclusion}
\chaptermark{Conclusion}
\label{chapter:conclusion}

\epigraph{\textit{The only truly secure system is one that is powered off, cast in a block of concrete and sealed in a lead-lined room with armed guards - and even then I have my doubts.}}{Gene Spafford}

\minitoc

%%%%%%%%%%%%%%%%%%%%%%%%%%%%%%%%%%%%%%%%%%%%%%%%%%%%%%%%%%%%%%%%%%%%%%%%%%%%%%%%%%%%%%%%%%%%%%%
\section{Synthesis}
With the rapid proliferation of embedded systems in safety-critical and high-reliability domains, ensuring their robustness against faults and attacks has become a pressing concern. While traditional protection efforts have focused on computation and storage elements such as CPUs and memory, this work highlights the critical and often overlooked vulnerabilities within on-chip communication architectures—especially system buses.

Through a systematic analysis of three widely used bus protocols—Wishbone, AXI-Lite, and AXI—this study identifies structural weaknesses that can be exploited by fault injection attacks, including those launched via electromagnetic interference. Experimental evidence demonstrates that such attacks can effectively bypass security mechanisms and compromise data integrity, even in systems with strong software-level protections.

To address these challenges, this research proposes a dedicated hardware-based countermeasure for the Wishbone bus architecture. Unlike existing software solutions, our approach integrates fault detection and response mechanisms directly into the bus infrastructure, achieving enhanced resilience without significant performance penalties. Comparative evaluations show that while software countermeasures offer flexibility, they are insufficient in isolation. The proposed hardware solution provides a more comprehensive and robust defense, particularly against persistent and low-level injection strategies.

Based on the discussion in Chapter 1: Attacks, this work establishes a systematic understanding of system-level threats targeting embedded electronic systems. By defining attacks as intentional actions aimed at compromising system security, the chapter structures the threat landscape into two primary categories: hardware-based and software-based attacks. Hardware attacks exploit the physical characteristics of devices through techniques such as reverse engineering, side-channel analysis, hardware Trojans, and fault injection, the latter being especially relevant due to its applicability across abstraction layers from gates to control logic. In contrast, software attacks operate at the logical level, manipulating code, control flow, or memory to trigger unintended behavior or escalate privileges.

A key takeaway from this chapter is the hierarchical nature of vulnerabilities, where low-level physical faults can manifest as high-level software anomalies. This insight reinforces the importance of analyzing security threats across multiple abstraction levels. Furthermore, the chapter surveys a variety of countermeasures—including redundancy, detection mechanisms, desynchronization, and isolation—highlighting the trade-offs between protection strength and implementation complexity. These foundational analyses guide the development of more effective, multi-layered defense strategies throughout the remainder of the thesis.

From Chapter 2: Experiment Setup, this thesis establishes a robust and flexible experimental framework for evaluating fault injection attacks on SoC interconnects. By utilizing LiteX as the base platform, we constructed customizable SoC instances with precise control over bus protocols, memory mapping, and peripheral configurations. This configurability, along with support for both simulation and physical deployment on FPGA platforms, enables scalable and reproducible testing. The framework is built around the VerifyPin benchmark suite, which includes a baseline implementation and seven variants with software countermeasures. VerifyPin simulates a secure PIN verification process, making it suitable for evaluating control flow and memory-level fault impacts.

To conduct systematic fault injection experiments, we adopted the FISSA toolchain, which integrates tightly with HDL simulators such as Questasim. FISSA automates the fault injection workflow, from script generation to simulation execution and log collection, thus enabling high-throughput testing. Within this setup, a realistic black-box attacker model is defined—one without knowledge of the correct PIN ("4321") and using an incorrect input ("0000") to ensure that all successful authentication results arise solely from fault-induced corruption. Faults are injected at the RTL level during the execution of the verifyPIN function, targeting control-related registers over multiple cycles.

To comprehensively evaluate system robustness, multiple fault models are considered. These include single-bit flips, unrestricted bit manipulations within a single register, dual-bit upsets (2-bit flips), and complex simultaneous faults across two distinct registers. Each model reflects practical fault conditions such as those caused by radiation or EM interference. The experimental configuration thus provides a comprehensive basis for analyzing vulnerability patterns and evaluating the effectiveness of both existing and newly proposed countermeasures in subsequent chapters.

From Chapter 3: Vulnerabilities Exploitation on the Bus, this thesis highlights the practical challenges and security implications of conducting fault injection attacks on SoC interconnects. By comparing the behavior of Wishbone, AXI-Lite, and AXI under different fault models, we identified critical vulnerability patterns, particularly in control logic and handshake mechanisms. Our findings emphasize the need for targeted protection of key control signals such as ACK and SEL, as their compromise can lead to severe disruptions including instruction skipping and unauthorized memory access.

The analysis demonstrates that while advanced bus protocols like AXI and AXI-Lite offer structural complexity that may deter simplistic attacks, they simultaneously introduce a broader and less transparent attack surface. In contrast, the Wishbone bus, though more vulnerable due to its simplicity, provides an ideal platform for controlled and precise experimentation. As a result, this study selects Wishbone as the primary target for countermeasure development, enabling focused and reproducible evaluation of protection strategies in the following chapters.

From Chapter 4: Protection Implementation on Wishbone Bus and Test, this thesis presents a comprehensive evaluation of both software and hardware countermeasures aimed at mitigating fault injection attacks on the Wishbone bus. Experimental evidence confirms that vulnerabilities are concentrated around key control signals—specifically the ACK, SEL, and grant registers. Structural weaknesses such as the lack of validation logic, combinational multiplexer paths, and absence of built-in error detection make these elements prime fault targets, capable of leading to successful authentication bypass or instruction-level disruption.

To address these vulnerabilities, the chapter first evaluates the effectiveness of seven software-level countermeasures embedded within the VerifyPin benchmark. While these approaches—ranging from loop counters to step verification—offer moderate resistance against certain attack types, they consistently fail to prevent even simple bit-flip faults when acting alone. This highlights the inherent limitations of software-only solutions, particularly their dependence on the same vulnerable execution context they are designed to protect.

In contrast, six hardware countermeasures—including parity checks, duplication, complementary duplication, Hamming code, SECDED, and triplication—were implemented and analyzed. The results show that hardware-based defenses significantly outperform software approaches, achieving near-zero or zero attack success rates across multiple fault models. Among them, duplication and triplication strategies proved most effective, especially when evaluated under the dual-register manipulation fault model. Furthermore, the overhead introduced by hardware logic remains negligible in terms of resource usage and execution time.

Despite the gains offered by standalone solutions, no individual hardware or software mechanism is capable of eliminating all vulnerabilities. To improve protection coverage, the study explores hybrid combinations of software and hardware countermeasures. While some configurations yield incremental improvements, persistent weaknesses under complex fault models—especially Manipulate Two Registers—underscore the need for more robust design strategies. As a result, a custom triplication-based, detection-only countermeasure is proposed and deployed to the ACK, SEL, and grant signals.

Final testing confirms the effectiveness of this proposed approach: across all conducted experiments using benchmark versions V0 and V7, fault injection results were confined to system crashes, silent failures, or successful detections—with no instances of unauthorized data access or control flow manipulation. These findings demonstrate that the proposed hybrid triplication mechanism offers complete coverage under the evaluated fault models and can serve as a practical, lightweight hardware-hardening strategy for future SoC designs.

Finally, to conclude this part, all the experiments were carried out on a server with the following configuration Xeon Gold 5220 (2.2~GHz, 18C/36T), 128~GB RAM, Ubuntu 20.04.6 LTS and Questasim 10.6e. We ran more than 40 million simulations for all our fault models, and each simulation took an average of 3.29 seconds to run on our server. This totally cost us more than 2 months.

%%%%%%%%%%%%%%%%%%%%%%%%%%%%%%%%%%%%%%%%%%%%%%%%%%%%%%%%%%%%%%%%%%%%%%%%%%%%%%%%%%%%%%%%%%%%%%%
\section{Perspectives}

While our study has successfully identified fault injection vulnerabilities on Wishbone, AXI-Lite, and AXI buses, the scope of implemented countermeasures in this thesis is limited to the Wishbone protocol. This restriction was made primarily for experimental control and reproducibility. However, to achieve a broader understanding of fault resilience across different bus architectures, future research should extend protection strategies—both hardware and software—to AXI-Lite and AXI buses. A comparative evaluation across these protocols would offer deeper insights into the interplay between bus complexity, control logic structure, and countermeasure effectiveness.

On the software side, this thesis primarily employed the VerifyPin benchmark as a representative testbed to analyze fault exploitation and evaluate the impact of software-level countermeasures. While VerifyPin provided a suitable baseline for controlled experimentation, it lacks the diversity and complexity required to generalize results across application scenarios. Future work should involve the implementation of a wider range of software countermeasures—such as control-flow integrity, runtime checks, and execution path verification—within the same SoC framework. These should be tested not only for fault detection capabilities but also for performance overhead, fault coverage, and compatibility with hardware-level protection.

Regarding hardware-based defense, the six protection strategies evaluated in this work—parity checking, duplication, complementary duplication, Hamming coding, SECDED, and triplication—offer a spectrum of trade-offs between complexity and fault tolerance. Nonetheless, numerous alternative mechanisms remain unexplored. For instance, hardware obfuscation, lightweight cryptographic primitives, error-detecting pipelines, and temporal redundancy could offer promising results when integrated into the SoC fabric. Future studies should implement these techniques on the same experimental structure and assess their robustness against realistic and complex fault models, as well as their impact on area, power, and timing constraints.

Finally, while our vulnerability analysis covered three mainstream interconnects—Wishbone, AXI-Lite, and AXI—many other bus architectures are widely deployed in embedded systems and deserve attention. Examples include AMBA AHB, Avalon, TileLink, and proprietary or customized buses used in domain-specific SoCs. Each of these protocols may exhibit unique structural weaknesses and response patterns under fault conditions. A systematic study of these alternatives would broaden the applicability of our findings and support the development of bus-agnostic or protocol-specific countermeasure frameworks.
%%%%%%%%%%%%%%%%%%%%%%%%%%%%%%%%%%%%%%%%%%%%%%%%%%%%%%%%%%%%%%%%%%%%%%%%%%%%%%%%%%%%%%%%%%%%%%%