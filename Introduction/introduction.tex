\chapter*{Introduction}
\addcontentsline{toc}{chapter}{Introduction}
\chaptermark{Introduction}

Embedded systems have become an integral part of modern technology, powering a wide range of applications from consumer electronics and industrial control to automotive systems and critical infrastructures. Their widespread adoption is driven by their ability to deliver dedicated functionality with efficiency, compact size, and relatively low cost. As these systems continue to proliferate and play essential roles in safety-critical and real-time applications, their reliability and security have become increasingly important concerns. Faults—whether caused by hardware defects, environmental conditions, or malicious interference—can lead to severe consequences, including system crashes, data corruption, or even catastrophic failures in mission-critical scenarios~\cite{bar2006sorcerer} \cite{karaklajic2013hardware}. Therefore, ensuring the correctness and resilience of embedded systems has emerged as a vital area of research within both academia and industry.

One crucial aspect of system resilience is the identification and mitigation of potential vulnerabilities. Traditional fault tolerance strategies address CPU-level execution errors and memory corruption,  For instance, the work in ~\cite{Shukla2023An} introduces a novel parity-based fault-tolerant instruction decoder architecture tailored for dual-core processors implemented on FPGAs. Similarly,\cite{Ramos2018Efficient} proposes a low-overhead, parity-based technique to protect the register file in RISC-V soft processors on Xilinx FPGAs, leveraging implementation redundancy and validating its efficacy through fault injection experiments. In another direction \cite{Medwed2011Arithmetic} presents efficient ALU architectures utilizing multi-residue codes to achieve high error detection capabilities. The proposed methods minimize error weight and are shown to have a minimal performance impact in typical smart card applications, with detailed trade off analysis in terms of area and critical path delay. Recent research has shown that other components—particularly the on-chip communication infrastructure such as system buses—are also susceptible to faults and attacks. Despite the vital role system buses play in facilitating data transfer between processing units, memory, and peripherals, they are often neglected in security and reliability discussions. This oversight constitutes a significant blind spot in the comprehensive protection strategy of embedded systems.

Moreover, as attack surfaces continue to expand, adversaries are becoming increasingly sophisticated, targeting non-traditional vectors such as fault injection and side-channel manipulation to compromise system integrity. Among these, bus-level attacks have attracted growing attention due to their ability to disrupt communication, intercept data in transit, or inject malicious payloads—often without requiring direct access to the CPU. In~\cite{mishra2024faults}, the authors demonstrate the use of electromagnetic pulse (EMP) injections to compromise the ARM TrustZone by targeting the bus. Unlike the work proposed in this manuscript, their solution relies on a software-based countermeasure, which is not evaluated for the potential introduction of new vulnerabilities. Furthermore, their work does not analyze overhead, execution time, or the increase in program complexity relative to the original implementation. Despite the central role of interconnect networks in System-on-Chip (SoC) architectures, both logical and physical attacks on these components remain underexplored. Addressing this critical gap is the primary objective of the present work.

This study advances the field by conducting an in-depth analysis of the three bus architectures~\cite{mitic2006survey}, systematically identifying potential vulnerabilities within the bus communication system (Wishbone, AXI-Lite, AXI). Furthermore, for the Wishbone bus, building upon prior work that demonstrates the feasibility of bus-level attacks~\cite{zhao2024communication}, we introduce a comprehensive hardware-based countermeasure specifically designed to defend against such threats, underscoring the insufficiency of software-only solutions. In addition, we perform a comparative analysis of existing countermeasures, illustrating the limitations of alternative approaches and demonstrating the superior effectiveness of our proposed hardware solution in ensuring robust and reliable system protection.

\section*{Contributions}
\addcontentsline{toc}{section}{Contributions}

This thesis makes the following key contributions to the advancement of secure System-on-Chip (SoC) design, particularly in the context of defending on-chip communication infrastructures against emerging fault injection and bus-level attacks:

\begin{enumerate}
\item Comprehensive analysis of bus architectures: We perform a detailed architectural and functional analysis of three widely used on-chip communication protocols—Wishbone, AXI-Lite, and AXI. By examining their transaction models, signal structures, and timing behaviors, we identify structural and design-level properties that may expose them to fault injection, bus snooping, and data hijacking attacks.

\item Vulnerability identification across multiple bus systems: Through systematic evaluation, we uncover previously underexplored vulnerabilities in each of the three bus systems. These include weak or absent integrity checks, unprotected data paths, and susceptibility to transient fault propagation—especially in unsecured or resource-constrained SoC environments.

\item Countermeasure evaluation for the Wishbone bus: We conduct a comparative study of existing countermeasures applicable to the Wishbone bus, including both software-based and hybrid techniques. Our analysis highlights the trade-offs between security coverage, performance overhead, and implementation complexity. We show that software-only approaches are insufficient for high-assurance applications due to limitations in fault detection granularity and real-time response.

\item Design and implementation of a hardware-based countermeasure: As a core contribution, we design a novel hardware-based countermeasure for the Wishbone bus that provides full-spectrum protection against bus-level attacks. Our approach integrates parity-based error detection, signal redundancy, and protocol monitoring logic to defend against both logical and physical attacks. Experimental results demonstrate that the proposed solution offers robust security with minimal performance and area overhead, making it suitable for FPGA-based and resource-constrained embedded systems.
\end{enumerate}

\section*{Organisation of document}
\addcontentsline{toc}{section}{Organisation of document}

This thesis is organized into five chapters. Chapter 1 introduces the context of system-level attacks in SoCs, defining hardware and software fault injection methods and presenting countermeasure classifications. Chapter 2 describes the experimental setup, including SoC construction using LiteX, bus configurations, fault models, benchmarks, and attacker assumptions. Chapter 3 focuses on vulnerability exploitation in the Wishbone bus, AXI and AXI-Lite buses. Chapter 4 evaluates both software and hardware countermeasures in the Wishbone bus, comparing their effectiveness, and proposing a custom hardware-based solution. Finally, Chapter 5 summarizes the main contributions and outlines directions for future work.

\section*{Publications}

During the course of the Ph.D., I actively presented my research at several international and national scientific events:
\begin{itemize}
    \item I delivered an oral presentation at the IEEE International Conference on Cyber Security and Resilience (CSR), held in London, UK, where I also published the peer-reviewed paper entitled “Communication Architecture Under Siege: An In-depth Analysis of Fault Attack Vulnerabilities and Countermeasures”. This work focused on fault injection vulnerabilities in SoC communication infrastructures and proposed a multi-layered countermeasure framework.
    \item I presented a poster at the JAIF 2023 (Journée thématique sur les attaques par injection de fautes), held in Gardanne, France. The poster showcased preliminary results on fault propagation behaviors across bus architectures, contributing to the national dialogue on fault injection resistance in embedded systems.
    \item I also participated in the GDR SoC 2025 Workshop in Lorient, France, where I gave a poster presentation summarizing recent advances in fault-tolerant communication protocols and discussed future integration into secure SoC designs.
\end{itemize}






